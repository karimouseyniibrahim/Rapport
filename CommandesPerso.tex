%%%%%%%%%%%%%%%%%%%%%%%%%%%%%%%%%%%%%%%%
%           Commandes perso            %
%%%%%%%%%%%%%%%%%%%%%%%%%%%%%%%%%%%%%%%%

\newcommand{\alp}{\texorpdfstring{\ensuremath{\upalpha}\xspace}{alpha }}
\newcommand{\bet}{\texorpdfstring{\ensuremath{\upbeta}\xspace}{b\'{e}ta }}
\newcommand{\alpbet}{\texorpdfstring{\ensuremath{\upalpha-\upbeta}\xspace}{alpha-b\'{e}ta}}
\newcommand{\alpt}{\ensuremath{\alpha_2}\xspace}
\newcommand{\strt}{\gls{strt}\xspace}


% Tenseur des déformation cylindrique
\newcommand{\epsrr}{\ensuremath{\varepsilon_{rr}}\xspace}
\newcommand{\epstt}{\ensuremath{\varepsilon_{\theta\theta}}\xspace}
\newcommand{\epszz}{\ensuremath{\varepsilon_{zz}}\xspace}
\newcommand{\epsrt}{\ensuremath{\varepsilon_{r\theta}}\xspace}
\newcommand{\epstz}{\ensuremath{\varepsilon_{\theta z}}\xspace}
\newcommand{\epszr}{\ensuremath{\varepsilon_{zr}}\xspace}

\newcommand{\matlab}{\textsc{Matlab}\texttrademark\xspace}



%% Figures centrées, et en position 'here, top, bottom or page'
\newenvironment{figureth}{%
		\begin{figure}[htbp]
			\centering
	}{
		\end{figure}
		}

		
%% Tableaux centrés, et en position 'here, top, bottom or page'
\newenvironment{tableth}{%
		\begin{table}[h]
			\centering
			%\rowcolors{1}{coleurtableau}{coleurtableau}
	}{
		\end{table}
		}


%% Sous-figures centrées, en position 'top'		
\newenvironment{subfigureth}[1]{%
	\begin{subfigure}[t]{#1}
	\centering
}{
	\end{subfigure}
}

\newcommand{\citationChap}[2]{%
	\epigraph{\og \textit{#1} \fg{}}{#2}
}

%% On commence par une page impaire quand on change le style de numérotation de pages 
\let\oldpagenumbering\pagenumbering
\renewcommand{\pagenumbering}[1]{%
	\cleardoublepage
	\oldpagenumbering{#1}
}

\newcommand{\mcd}{model checking distribué}
\newcommand{\mc}{model checking}
\newcommand{\border}{\emph{Border}}
\newcommand{\ssti}{Search\_ States\_ To\_ Impor}
\newcommand{\notifier}{\emph{Notifier}}
\newcommand{\bn}{\emph{\border{} et \notifier{}}}

\newcommand{\ei}{Etat Interne}
\newcommand{\BbSSD}{RSC}
\newcommand{\ee}{Etat Externe}
\newcommand{\deplacer}{déplacer}
\newcommand{\dupliquer}{dupliquer}
\newcommand{\parametreone}{\alpha}
\newcommand{\parametretwo}{\beta}
\newcommand{\parametretree}{\gamma}
\newcommand{\parametrefive}{\delta}
\newcommand{\parametrefour}{\gamma m}
\newcommand{\ministere}{R\'{e}publique  Alg\'{e}rienne D\'{e}mocratique et Populaire Minist\'{e}re de  l'enseignement Sup\'{e}rieur et de la Recherche Scientifique Faculté des Nouvelles Technologies de l'Information et la Communication Département d’Informatique Fondamentale et ses Applications}
\newcommand{\sneuf}{\og S9 \fg{}}
\newcommand{\s}[1]{\og #1 \fg{}}
\newcommand{\mone}{\og machine 1 \fg{}}
\newcommand{\mi}{\og machine i \fg{}}
\newcommand{\mj}{\og machine j \fg{}}
\newcommand{\mtwo}{\og machine 2 \fg{}}
\newcommand{\mtree}{\og machine 3 \fg{}}
\newcommand{\curenti}{curent_{iteration}}
\newtheorem{Exemple}{Exemple}[section]
\newtheorem{theorem}{Theorem}
\newtheorem{definition}[theorem]{Définition}
\newtheorem{etape}[theorem]{Étape}

\newcommand{\mysection}[2][]
   {\section[#1]
     {\centering #2}
       \setcounter{figure}{0}
       \renewcommand{\thefigure}{\thesection.\arabic{figure}} %Arabic figures
       \renewcommand{\thesection}{\Roman{section}}               %Roman numeral title
}
\newcommand{\mysectionNoNumerotation}[2][]
   {\section[#1]
     {#2}
       \renewcommand{\thesection}{{section}}               %Roman numeral title
}
\setcounter{secnumdepth}{3} 
%Subsections do not get centered
\allowdisplaybreaks
\newcommand{\mysubsection}[2][]
{\renewcommand*{\thesubsubsection}{\alph{subsubsection}}
\subsubsection[#1]{#2} 
}
\newcommand{\sectionbreak}{\clearpage}
\algnewcommand\algorithmicforeach{\textbf{for each}}
\algdef{S}[FOR]{ForEach}[1]{\algorithmicforeach\ #1\ \algorithmicdo}

\SetNlSty{}{}{}
\let\oldnl\nl% Store \nl in \oldnl
\newcommand\nonl{%
  \renewcommand{\nl}{\let\nl\oldnl}}% Remove line number for one line
