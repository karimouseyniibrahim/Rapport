
\section{Politique de distribution de l’espace d’états basée sur le	comportement du système}
 
La politique proposée, vise à optimiser la distribution de l’espace d’états ainsi que le temps de vérification du model checking. Pour un système donné spécifiée à partir d'un réseau de Petri, nous utilisons un algorithme de génération parallèle pour construire la structure de Kripke distribuée, en utilisant la fonction \emph{MD5} pour le partitionnement de l’espace d’états entre les machines du réseau. Après cela, lors de l'exécution du model checking, chaque machine analyse son fragment (structure de Kripke partielle) et élabore certaines statistiques sur les états par rapport à la vérification. Les statistiques  générés sur chaque état mesure la dépendance de l'état par rapport aux états de la machine locale ou aux états des machines distantes. Une fois le processus de vérification du model checking est terminé, le protocole de redistribution peut être lancé afin de calculer l'ensembles des états à déplacer et à dupliquer pour chaque état \emph{Externe} et \emph{sollicité}. Après le calcul des ensembles on décide, sur la base de l'écart minimum et maximum des états à stocker sur la machine, si l'ensemble peut être déplacé ou resté dans une machine afin de minimiser la taux de communications entre les machines tout en garantissant une bonne distribution. Ainsi, une bonne distribution n'implique pas la réduction des transitions reliant les parties.


L’algorithme général est l'Algorithme \ref{bbssd} :\\
\begin{algorithm}[H]\label{bbssd}
	\SetAlgoLined
	\SetKwIF{If}{ElseIf}{Else}{if}{then}{else if}{else}{endif}
	 Générer la structure de Kripke distribuée à partir de la spécification d'un réseau de Petri\;
	 Exécution du model checking et élaboration des statistiques sur chaque état\;
	 Redistribuer les états en optimisant les performances du système\;	  
	\caption{\CDS{}}
\end{algorithm}
Dans les sections suivantes, nous détaillons les phases de cet algorithme.

 
\begin{figure}[h]
	\centering
	\includegraphics[height=3.0in,width=1.0\textwidth]{img/protocole}
	\caption{Schéma de la politique de redistribution de l'espace d’états}
\end{figure}
	
 

\subsection{Génération distribuée de l’espace d’états à partir de la spécification d'un réseau de Petri}{
La spécification du réseau de Petri est faite à partir d'une machine choisie aléatoirement. Le processus de génération distribuée de l'espace d'états est fait par l’exploration de l’état initial en générant tous ses états successeurs. Par la suite, toutes les machines disponibles sur le réseau contribuent à la construction des fragments de l’espace d’états distribué. Pour chaque nouvel état généré appartenant à une machine $M_i$, tous ses états successeurs sont générés. Un état successeur peut être dans la même machine ou dans une machine distante. Chaque machine $M_i$ envoie tous ses états externes aux machines déterminées par la fonction de partition. La fonction de partition est basée sur la fonction de hachage cryptographique \emph{MD5} qui renvoie un index $j\;(j \in 0, N-1)$. La fonction de hachage adoptée réalise un bon équilibrage de charge entre les $N$ machines du réseau.  La génération distribuée se termine lorsqu'il n'y a aucun états en attente d'être exploré.\\
L’algorithme de génération distribuée de l’espace d’états est l'Algorithme \ref{gendistribuer} :\\
\SetKwFunction{printlcs}{}
\begin{algorithm}
	\SetAlgoLined
	\SetKwIF{If}{ElseIf}{Else}{if}{then}{else if}{else}{endif}
	\SetAlgoLined	
	\SetKwInOut{Mi}{$M_i$}
	\SetKwInOut{Mj}{$M_j$}
	\SetKwInOut{etati}{$Etat_i$}
	\SetKwInOut{ei}{$E_i$}
	\SetKwInOut{si}{$S_i$}
	\SetKwInOut{t}{$t$}
	\SetKwInOut{T}{$T$}
	\SetKwInOut{smm}{$s,m,m'$}
	\SetKwInOut{tb}{$t_b$}
	\SetKwInOut{sm}{$s.m$}
	\SetKwInOut{sprp}{$s.L$}
	\SetKwInOut{te}{$TExplore$}
	\SetKwInOut{tr}{$TR_i$}
	\SetKwInOut{Pres}{$Pres$} 
	\SetKwInOut{Post}{$Post$}
	\SetKwInOut{pps}{$Prop\_Pres$}
	\SetKwInOut{ppt}{$Prop\_Post$}
	\AlgoDontDisplayBlockMarkers\SetAlgoNoEnd\SetAlgoNoLine%
	\Mi{machine i}
	\Mj{machine j}
	\etati{\{dehors,dedans\}}
	\ei{init à $\emptyset$, pile des états non encore explorés appartenant à la machine i}
	\si{init à $\emptyset$, la liste des états déjà explorés appartenant à la machine i }
	\smm{un état}
	\sm{marquage d'un états}
	\sprp{liste des propriétés d'un états}
	\tr{la liste des relations de transitions de la machine i}
	\T{a liste de transitions}
	\tb{transition bloquant}
	\te{liste transition admissible}
	\t{une transition}
	\Pres{matrice pres du réseau de Petri}
	\Post{matrice post du réseau de Petri}
	\pps{matrice pres des propriétés}
	\ppt{matrice post des propriétés}

	 \caption{Génération Initiale Distribuée}\label{gendistribuer}
 
	\SetKwFunction{algo}{}\SetKwFunction{proc}{proc}
	\Begin{	 
	 \SetKwProg{myalg}{Reception Etat}{}{}
	 
	 \myalg{\algo{m:etat}envoyé par $M_j$}{  
	 	$s\leftarrow findByMarquage(m,S_i)$\;	
	 	\uIf{($ s ==null$)}{
	 		$m.sub\leftarrow\{ M_j \}$\;
	 		Empiler(m,$E_i$)\;	 		
	 		\If{($Etat_i$!= dedans)}{
	 			GenerationDistribue$_i$()\;
 			}
	 	}
 		\Else{
 		   ajouter $M_j$ à la liste des machines de l'etat portant identifiant \emph{m} \;			
 		}	 		  		
	 }{} 	  	
   	\SetKwProg{myalgone}{GenerationDistribue$_i$}{}{}   
  	 \myalgone{\algo{}}{  
	  	$Etat_i\leftarrow dedans$\;	
	  	\While{($ E_i !=\emptyset$)}{
	  		$m\leftarrow depiler(E_i)$\;
	  		$TExplore \leftarrow \{t\in T\mid m.m[t>\}$\;
	  		$S_i\leftarrow S_i \cup \{m\}$\;
	  		\ForEach{($t \in TExplore$)}{
	  			$m'.m\leftarrow m.m + Post(t)-Pres(t)$\; 
	  			$m'.L\leftarrow m.L + Prop\_Post(t)-Prop\_Pres(t)$\; 
	  			$M_j \leftarrow MD5(m'.m)$\;
	  			\uIf{$M_j = =M_i$}{
	  				$s\leftarrow findByMarquage(m,S_i)$\;
	  				\If{(findByMarquage(m'.m,$S_i$)==null and findByMarquage(m'.m,$E_i$)==null)}{
	  					Empiler(m',$E_i$)\;
	  				}
  				}\Else{  				
  					Envoyer (m') à $M_j$\;
  				}
  				$TR_i\leftarrow TR_i\cup \{(m,t,m')\}$	\;
  			}	
	  			
	  		\If{(TExplore== $\emptyset$)  }{
	  		 	$TR_i\leftarrow TR_i\cup \{(m,t_b,m)\}$\;
	  		}
  		}
	  	$Etat_i\leftarrow dehors$\;
	  }{}
  }
\end{algorithm}
\begin{algorithm}
	\LinesNumbered
	% This is to restore vline mode if you did not take the package as \usepackage[linesnumbered,ruled,vlined]{algorithm2e}
	\SetAlgoVlined 
 	 \SetKwProg{myalgone}{Initialisation}{}{ }
  	 \myalgone{\algo{}}{  
  	 $T\leftarrow init$\;
  	 $Pres\leftarrow init$\;
  	 $Post\leftarrow init$\;
  	 $Prop\_Pres\leftarrow init$\;
  	 $Prop\_Post\leftarrow init$\;
  	 $m.m\leftarrow init$\;
  	 $m.prop\leftarrow init$\;
  	 $M_i \leftarrow MD5(m.m)$\;
  	 Envoyer (m) à $M_j$\;
  
  }
	
\end{algorithm} 
}
\pagebreak
%%%%%%%%%%%%%%%%%%%%%%%%%%%%%%%%%%%%%%%%%%%%%%%%%%%%%%%%%%%%%%%%%%%%%%%%%%%%%%%%%%%%%%%%%%%%%
%%									section 1.	Principe Mod\`{e}le Checker Distribu\'{e}											%
%%%%%%%%%%%%%%%%%%%%%%%%%%%%%%%%%%%%%%%%%%%%%%%%%%%%%%%%%%%%%%%%%%%%%%%%%%%%%%%%%%%%%%%%%%%%%

\section{Principe Model Checking Distribu\'{e}}
 
Le model checking distribu\'{e} propos\'{e} par Bouneb Zine permet une v\'{e}rification parall\`{e}les d’une formule $\varphi$  sur les fragments de la structure de Kripke distribu\'{e}e. Du fait de la distribution, la v\'{e}rification de $\varphi$ de certain \'{e}tat \emph{S} d\'{e}pend de la v\'{e}rit\'{e} en \emph{S’} de cette formule, \emph{S’}  h\'{e}berg\'{e} dans une  machine distante $M_j$. La machine $M_i$ effectue la v\'{e}rification de $\varphi$ sur le fragment d\'{e}tenue en consid\'{e}rant la valeur logique de \emph{S’} ind\'{e}cidable  $L (\emph{S’},\varphi )=\perp$ car la formule $\varphi$ peut \^{e}tre v\'{e}rifier en \emph{S’}. Lorsque la machine $M_j$ termine le calcul est que la formule est v\'{e}rifi\'{e}e en \emph{S’}, aucune notification n'est envoy\'{e}e \`{a} la machine $M_i$, la machine $M_i$ consid\`{e}re que la formule est v\'{e}rifi\'{e}e en \emph{S’}, sinon la machine $M_j$ envoi \`{a} la machine $M_i$ la valeur logique de \emph{S’} qui est L (\emph{S’}, $\varphi$) =false. Dans ce cas, $M_i$ reprend le calcul avec la valeur envoyée pour d\'{e}terminer la v\'{e}rit\'{e}e des \'{e}tats prédécesseurs. Une fois que la terminaison est d\'{e}tect\'{e}e  la machine qui détient l’\'{e}tat initial d\'{e}duit la v\'{e}racit\'{e}e de la formule $\varphi$.

A ce principe est ajouté l'étiquetage des \'{e}tats pendant l'exécution de la de v\'{e}rification. Les \'{e}tiqu\`{e}tes associées aux \'{e}tats permettent de savoir si la valeur logique d'un état dépend de ses successeurs, ainsi que le nombre de calcul permettant de déterminer la valeur logique de l'état. A partir de ces informations il est possible  de redistribuer l'espace d’\'{e}tats afin acc\'{e}l\'{e}rer le temps du calcul de la vérification.

\begin{Exemple}
   Une structure de Kripke distribu\'{e} sur 3 machines.
   \centering
	\includegraphics[height=2in]{img/skd1.png}
	
	\captionof{figure}{Structure de kripke distribu\'{e}}

\end{Exemple}

Les résultats observés durant la vérification de la formule \s{AG(a)} sont repr\'{e}sent\'{e}s dans le Tableau \ref{statistique}, les lignes repr\'{e}sentent les donn\'{e}es concernant l'\'{e}tat et les colonnes les types de donn\'{e}es construits durant le processus de v\'{e}rification. Les abr\'{e}viations utilis\'{e}es sont:
\begin{description}[,leftmargin=2cm,labelindent=1em]
	\item[Id :] L'identifiant de l’état,
	\item[T  :] Le type d’\'{e}tat (I : interne ; E : externe),
	\item[Site :] La machine distante détenant l'\'{e}tat (Externe),
	\item[RF :] La raison de la non validit\'{e} de la formule (c : che$M_i$n/ e : \'{e}tat),
	\item[i :] Le nombre de recalcul permettant de d\'{e}ter$M_i$ner la valeur logique,
	\item[VL :] La valeur logique de la formule à vérifier $(true : 1 ; false : 0 ; \perp : -1)$.
\end{description}

\begin{tableth}
	\centering
	\begin{tabular}{|*{14}{c|}}
		\hline
		\multicolumn{7}{|c|}{It\'{e}ration 1}&\multicolumn{7}{|c|}{It\'{e}ration 1}\\
		\hline
		Id&	T	&VL&	i	&Site&	M&	ME&Id&	T	&VL&	i	&RF&	M&	Site\\
		\hline
		S1	&I	&0&	1	&E	&M1	&&&&&&&&\\
		\hline
		S2	&I	&-1&	1	&M1	&&&&&&&&&\\
		\hline
		S3	&E	&-1	&1		&M1	&M3&&&&&&&&\\
		\hline
		S4	&E	&-1	&1		&M1&	M2&&&&&&&&\\
		\hline
		S2& 	E	&-1	&1	&	M2&	M1&&&&&&&&\\
		\hline
		S4	&I&	-1	&1	&	M2	&&&&&&&&&\\
		\hline
		S4	&E	&-1&	1		&M3	&M2&&&&&&&&\\
		\hline
		S5	& I&	-1 &1	&	M3	&	&S5	&I	&0	&2&	C	&M3	&&\\
		\hline
		S3&	I	&-1	&1&		M3&&&&&&&&&\\
		\hline
		S1	&E	&-1	&1	&E	&M3&	M1&	S1	&E	&0	&2	&E	&M3&	M1\\
		\hline
	\end{tabular}
	\caption{Statistique des \'{e}tats}\label{statistique}
\end{tableth}

Apr\`{e}s la premi\`{e}re it\'{e}ration l'algorithme détecte sur la machine $M_1$ que la formule n’est pas v\'{e}rifi\'{e}e sur l’\'{e}tat \s{S1}, cela permet d'envoyer une notification à la machine $M_3$ qui possède un prédécesseur direct de cet \'{e}tat. L'algorithme sur la machine $M_3$  prend en compte la modification et relance le processus de v\'{e}rification sur le fragment. La valeur logique de l’\'{e}tat \emph{S5} bascule d'indécidable à false à la deuxi\`{e}me it\'{e}ration. Apr\`{e}s cette it\'{e}ration, aucune action de modification n'est détectée, sur la machine $M_3$ l'algorithme d\'{e}duit la v\'{e}rit\'{e} de la formule.
%%%%%%%%%%%%%%%%%%%%%%%%%%%%%%%%%%%%%%%%%%%%%%%%%%%%%%%%%%%%%%%%%%%%%%%%%%%%%%%%%%%%%%%%%%%%%
%%									section .	2.	Param\`{e}tre d’optimisation									%
%%%%%%%%%%%%%%%%%%%%%%%%%%%%%%%%%%%%%%%%%%%%%%%%%%%%%%%%%%%%%%%%%%%%%%%%%%%%%%%%%%%%%%%%%%%%%

\section{Param\`{e}tre d’optimisation}
 
Apr\`{e}s l'analyse du principe pr\'{e}c\`{e}dent on constate que la distribution des \'{e}tats li\'{e}s entrainent un nombre de calculs assez élevé pour déterminer la valeur logique de ces états. Le regroupement de ces états am\'{e}liore le temps de traitement. Ainsi pour minimiser la distribution des \'{e}tats li\'{e}s un certain nombre d'objectifs sont à prendre en consid\'{e}ration pour assurer  que l'espace des \'{e}tats est distribuée car la centralisation de ces états résous entièrement le probl\`{e}me. Les objectifs sont:

\begin{itemize}[leftmargin=*,labelindent=1em]
	\item Le temps de traitement à minimiser autant que possible ;
	\item Assurer l’\'{e}quilibrage de charge entre les machines ;
	\item Un nombre de dupliquâts réduit sur les machines.
\end{itemize}

Pour atteindre ces objectifs les paramètres $\parametreone{}$, $\parametretwo{}$ et $\parametretree{}$ sont utilisés. La prise en compte simultanée des trois param\`{e}tres permet de minimiser à la fois le temps de traitement et garantir l’\'{e}quilibrage de charge avec un nombre de dupliquâts réduit. La r\'{e}solution de ce problème entraine l'utilisation de fonctions objectives pour trouver l'espace de solution possible car il peut existé une infinit\'{e} de configurations respectant ces contraintes. Ainsi, diff\'{e}rente techniques sont offertes pour l'obtention de l'espace des solutions, dans ce contexte la technique d’\'{e}quilibre de Nash combin\'{e}e aux fonctions objectives est envisageable.

%%%%%%%%%%%%%%%%%%%%%%%%%%%%%%%%%%%%%%%%%%%%%%%%%%%%%%%%%%%%%%%%%%%%%%%%%%%%%%%%%%%%%%%%%%%%%
%%									section :	Principe d’\'{e}quilibre de Nash					%
%%%%%%%%%%%%%%%%%%%%%%%%%%%%%%%%%%%%%%%%%%%%%%%%%%%%%%%%%%%%%%%%%%%%%%%%%%%%%%%%%%%%%%%%%%%%%

 \subsection{Principe de l’\'{e}quilibre de Nash}
 
L'\'{e}quilibre de Nash est une solution propos\'{e}e par John Forbes Nash en 1950 \citep{depriester1950johnnash} pour la recherche d’une solution optimale. Il est couramment utilis\'{e} en th\'{e}orie des jeux. Un jeu pr\'{e}sente une combinaison de d\'{e}cisions individuelles, appel\'{e}es «strat\'{e}gies», où chaque joueur anticipe correctement les choix des autres; Il y a autor\'{e}alisation, puisque l'issue r\'{e}alis\'{e}e est le fruit de d\'{e}cisions prises en pensant qu'elle va se r\'{e}aliser. En th\'{e}orie des jeux la question que se pose un joueur au moment de faire son choix est : que va faire l'autre ? Ses croyances concernant le comportement des autres joueurs ont donc un rôle essentiel au moment de la d\'{e}cision. La diversit\'{e} de croyance correspond ainsi \`{a} une multiplicit\'{e} d'\'{e}quilibres. Dans les jeux coop\'{e}ratifs on autorise la communication et les accords entre joueurs avant la partie, les messages formul\'{e}s par un joueur sont transmis sans modification \`{a} l'autre joueur, les accords entre joueurs seront respect\'{e}s, ces hypoth\`{e}ses permettent d'obtenir un \'{e}quilibre de Nash.

En utilisant le principe de Nash, J.A D\'{e}sid\'{e}ri \citep{depriester2007jeanantoine} propose un algorithme de partitionnement de territoire qui se base sur des fonctions objectifs. Il consid\`{e}re le cas où on dispose des fonctionnel pr\'{e}pond\'{e}rantes, ainsi l'algorithme cherche \`{a} optimiser les fonctions prioritaires avec une moindre d\'{e}gradation tout en associant aux fonctionnelles secondaires des param\`{e}tres qui engendrent de grandes variations. La recherche de l’\'{e}quilibre de Nash se fait par \'{e}change de r\'{e}sultats obtenus pour chaque fonction objectif travaillant avec une partie seulement des variables, les autres \'{e}tant fix\'{e}es par les r\'{e}sultats obtenus pour les autres fonctions objectifs. Cet \'{e}quilibre est atteint quand l’optimisation de chaque fonction conduit toujours \`{a} la m\^{e}me solution.

En s'inspirant de ces principes, on propose une strat\'{e}gie d'optimisation des param\`{e}tres définie ci-dessus.
%%%%%%%%%%%%%%%%%%%%%%%%%%%%%%%%%%%%%%%%%%%%%%%%%%%%%%%%%%%%%%%%%%%%%%%%%%%%%%%%%%%%%%%%%%%%%
%%									section 4.	Principe de Optimisation (α, \beta, \gamma)								%
%%%%%%%%%%%%%%%%%%%%%%%%%%%%%%%%%%%%%%%%%%%%%%%%%%%%%%%%%%%%%%%%%%%%%%%%%%%%%%%%%%%%%%%%%%%%%

\subsection{Principe d'Optimisation de $\alpha, \beta\; et\; \gamma$}
 
Dans le cadre du model checking, les états présentent des informations par rapport à la vérification. L’analyse de ces résultats, montre que la distribution des états fortement liés augmente le temps de traitement de la vérification. A titre d'exemple sur l'exemple précédente, la distribution des états $S1$ et $S5$ a augmenté le temps de la vérification du model checking car ces états sont liés. Dans un exemple plus complexe o\`{u} un ensemble d’états liés sont distribués, le temps de la vérification sera très élevé. Pour avoir une distribution de l'espace d'états respectant les objectifs fixés nous optimisons les paramètres $\alpha, \beta \; et\; \gamma$ en simulant un jeu non coopératif entre les machines. L'optimisation de ces paramètres est faite comme suit:
\begin{itemize}
	\item Sur Chaque machine on recherche le nombre d'états liés à chaque état sur lequel la formule n'est pa vérifiée, le paramètre $\beta$ pour cet état est égale à ce nombre. Les états local qui disposent des prédécesseurs directs sur d'autres machines distantes, la valeurs du paramètre correspond aux nombres des successeurs directs et indirects liés au même états.
	\item La limite des états liés ou dépendant peut entraîner la duplication de certain états, ce nombre états correspond à la valeur du paramètre $\gamma$. 
	\item En appliquant une heuristique sur les paramètres $\beta \; et\; \gamma$ de deux machines qui sont liées par un état nous obtenons une optimisation du paramètre $\alpha$.
\end{itemize}
Dans les sections suivantes, nous détaillons les phases de ce principe d'optimisation, la structure de Kripke présentée sur la Figure \ref{skd2} sera utilisée dans les exemples. \\

A partir du nombre d'états stockés dans chaque machine il est possible de calculer le nombre minimum et maximum d’états susceptibles d’être stockés par deux machine. Le calcul de cet interval est comme suit :

\begin{itemize}
	\item  	Soit $(\beta1 \; et\; \gamma1)$ les valeurs de la machine $M_1$ respectivement pour la machine $M_2$ $(\beta2 \;et\; \gamma2)$.
	\item  	$\overline{\beta}$ la moyenne des états $\overline{\beta}=\frac{\beta1+\beta2}{2}$.
	\item  	L’écart type $\delta=\sqrt{\frac{\displaystyle\sum_{i=1}^{2}\beta_i^2 }{2}-\overline{\beta}^2} $
\end{itemize}
 
	\begin{center}
		\includegraphics[height=4in]{img/skd2.png}	
	\captionof{figure}{Structure de Kripke distribu\'{e}} \label{skd2}
	\end{center}

