%%%%%%%%%%%%%%%%%%%%%%%%%%%%%%%%%%%%%%%%%%%%%%%%%%%%%%%%%%%%%%%%%%%%%%%%%%%%%%%%%%%%%%%%%%%%%
%%									section :	Principe d’\'{e}quilibre de Nash					%
%%%%%%%%%%%%%%%%%%%%%%%%%%%%%%%%%%%%%%%%%%%%%%%%%%%%%%%%%%%%%%%%%%%%%%%%%%%%%%%%%%%%%%%%%%%%%

 \subsection{Principe de l’\'{e}quilibre de Nash}
 
L'\'{e}quilibre de Nash est une solution propos\'{e}e par John Forbes Nash en 1950 \citep{depriester1950johnnash} pour la recherche d’une solution optimale. Il est couramment utilis\'{e} en th\'{e}orie des jeux. Un jeu pr\'{e}sente une combinaison de d\'{e}cisions individuelles, appel\'{e}es «strat\'{e}gies», où chaque joueur anticipe correctement les choix des autres; Il y a autor\'{e}alisation, puisque l'issue r\'{e}alis\'{e}e est le fruit de d\'{e}cisions prises en pensant qu'elle va se r\'{e}aliser. En th\'{e}orie des jeux la question que se pose un joueur au moment de faire son choix est : que va faire l'autre ? Ses croyances concernant le comportement des autres joueurs ont donc un rôle essentiel au moment de la d\'{e}cision. La diversit\'{e} de croyance correspond ainsi \`{a} une multiplicit\'{e} d'\'{e}quilibres. Dans les jeux coop\'{e}ratifs on autorise la communication et les accords entre joueurs avant la partie, les messages formul\'{e}s par un joueur sont transmis sans modification \`{a} l'autre joueur, les accords entre joueurs seront respect\'{e}s, ces hypoth\`{e}ses permettent d'obtenir un \'{e}quilibre de Nash.

En utilisant le principe de Nash, J.A D\'{e}sid\'{e}ri \citep{depriester2007jeanantoine} propose un algorithme de partitionnement de territoire qui se base sur des fonctions objectifs. Il consid\`{e}re le cas où on dispose de fonctionnelles pr\'{e}pond\'{e}rantes, ainsi l'algorithme cherche \`{a} optimiser les fonctions prioritaires avec une moindre d\'{e}gradation tout en associant aux fonctionnelles secondaires des param\`{e}tres qui engendrent de grandes variations. La recherche de l’\'{e}quilibre de Nash se fait par \'{e}change de r\'{e}sultats obtenus pour chaque fonction objectif travaillant avec une partie seulement des variables, les autres \'{e}tant fix\'{e}es par les r\'{e}sultats obtenus pour les autres fonctions objectifs. Cet \'{e}quilibre est atteint quand l’optimisation de chaque fonction conduit toujours \`{a} la m\^{e}me solution.

En s'inspirant de ces principes, on propose une strat\'{e}gie d'optimisation des param\`{e}tres définie ci-dessus.