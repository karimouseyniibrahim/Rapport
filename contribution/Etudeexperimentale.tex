\section{Étude expérimentale}
Cette section présente les résultats expérimentaux obtenus en appliquant la méthode de redistribution proposée dans ce chapitre.

\subsection{Configuration de la mise en œuvre}
Nous avons développé l’outil qui génère une structure de Kripke distribuée. L’outil dispose d’un éditeur de graphique pour dessiner et modifier les systèmes analysés à partir d'un réseau de Petri (Figure \ref{apppetrinet}).

L’approche proposée a été implémenter avec le langage de programmation JAVA sur IDE IntelliJ, en utilisant le base de donnée orienté graphe Neo4J pour le stockage de l'espace d'états, le Docker pour l'hébergement de ces bases de données. le framework Java Agent Development est aussi utiliser pour faciliter la communication entre les machines.