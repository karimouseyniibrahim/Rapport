
\begin{procedure}
	\setcounter{AlgoLine}{0}
	\SetAlgoLined
  \Begin { 	
\nl  		\If{$size(states)\ne 0$}{\label{prline1}
 \nl 			add(states) in  Sj \;\label{prline2}
 \nl 		}
 \nl 		\If{$MG\ne null)$}{\label{prline3}
 \nl 			$send(element(MG),null)  to MG.id$\;\label{prline4}
\nl  		}
\nl  		algorithm \ref{alg6}()\;\label{prline6}
}
\caption{$Receive\_ Request(states :List of state,MG:state)$}
\end{procedure}

Les instructions décrivant la procédure ($Receive\_ Request$) sont expliquées comme suit:
\begin{description}
\item[ligne \ref{pline1} :] L'instruction de cette ligne vérifie que des états ont été  envoyés. La vérification de cette condition permet d'exécuter l'instruction de la ligne \ref{prline2}, elle permet de rajouter les états reçus à la structure de Kripke(ligne \ref{pline2}). 
\item[ligne \ref{pline3} :] L'instruction de cette ligne vérifie qu'une demande concernant un ensemble des états a été envoyé. La vérification de cette condition permet d'envoyer l'ensemble des états concernant la demande (ligne \ref{pline4}). 
\item[ligne \ref{pline5} :] L'instruction sur cette ligne permet de relancer le protocole de redistribution afin de rechercher un ensemble des états à déplacer.   
\end{description}