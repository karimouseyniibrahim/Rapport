\begin{function}[h]




\Begin{
\nl	 $MG\longleftarrow null$\;\label{lineI0}\\
\nl	  \ForEach{$e\in \; Border(F)$}{ \label{linerI1}
\nl					\ForEach{$e'\in \; Notifier(F) $}{\label{linerI2}
\nl			\For{$j=1$ \KwTo $size(e'.site)$}{\label{linerI3}
\nl				\If{($(e\in e'.f) and (exist(e.site[j]))and(e.\parametretwo{}>e.site[j].\parametretwo{}) 
					and$\;$ ((e'.\parametretwo{}>e'.site[j].\parametretwo{})or(e'.\parametretwo{}==e'.site[j].\parametretwo{} and e'.\parametretree{}>e'.site[j].\parametretree{})) and$\; $(((MG.\parametretwo{}>e'.site[j].\parametretwo{})or(MG.\parametretwo{}==e'.site[j].\parametretwo{}\; and\; MG.\parametretree{}>e'.site[j].\parametretree{})or \;MG ==null)$) 
					)}{\label{linerI4}
\nl					$MG\longleftarrow e'.site[j]$\;\label{linerI5}
				}
			}
								
		}
	}
\nl	\If{$MG==null$}{\label{linerI6}
		\nl\ForEach{$e\in \; \{Border(F)\cup Notifier(F)\}$}{\label{linerI7}
\nl			\ForEach{$site\in e.site $}{\label{linerI8}
\nl				\If{$((e.\parametretwo{}>site.\parametretwo{} )or (e.\parametretwo{}==site[j].\parametretwo{}\; and\; e.\parametretree{}>site.\parametretree{} )) and(MG ==null or ((MG.\parametretwo{}>site.\parametretwo{} )or \;(MG.\parametretwo{}==site[j].\parametretwo{} \;and\; e.\parametretree{}>site.\parametretree{} )))$}{\label{linerI99}
\nl					$MG\longleftarrow e'.site[j]$\;\label{linerI9}
				}
				
			}	   	
\nl			}\label{linerI10}
	}	 	
	\nl \Return MG\label{linerI11} 
}
	\caption{$Search\_ Min\_ States\_DeplaceOnDistantMachine$():state}
\end{function}


Les instructions décrivant la fonction précédente sont expliquées comme suit:
\begin{description}
	\item[ligne \ref{lineI0} :] L'instruction de cette ligne sert à initialiser la variable stockant les informations du couple minimal.
	\item[ligne \ref{linerI1} :] Parcourir les états \s{\border{}} sur lesquels la formule n'es pas vérifiée. Ainsi, pour chaque état parcouru, les états \s{\notifier} sont parcourus à la ligne \ref{linerI2} pour rechercher le couple minimal parmis les valeurs envoyées par les machines distantes au niveau des états en dépendance mutuelle. Le parcours des valeurs envoyées par les machines distantes est fait à la ligne \ref{linerI3}. Ainsi la vérification du couple minimal est faite  à la ligne \ref{linerI4}. Lorsqu'un couple minimal est trouvé le contenue de la variable \s{MG} est alors remplacé par les informations du nouveau couple (ligne \ref{linerI5}).
	\item[ligne \ref{linerI6} :]  L'instruction de cette ligne vérifie qu'aucun minimum n'a été trouvé lors de la recherche précédente. Lorsque la condition est vérifiée les instructions de la ligne \ref{linerI7} jusqu'à la ligne \ref{linerI10} sont alors exécutées.
	\item[ligne \ref{linerI7} :] L'instruction de cette ligne permet de parcourir les états \s{\bn{}} sur lesquels la formule n'es pas vérifiée.
	\item[ligne \ref{linerI8} :]  Pour chaque état, l'instruction de cette ligne parcoure les valeurs calculées par les machines distantes et recherche le couple minimum. La vérification du minimum est faite à la ligne \ref{linerI99}. Ainsi la vérification de cette condition entraine l'exécution de l'instruction de la ligne \ref{linerI9}, cela permet à remplacer l'ancien couple par le nouveau.
	\item[ligne \ref{linerI11} :] L'instruction écrite sur cette ligne permet de retourner les informations du couple minimal.
\end{description}

