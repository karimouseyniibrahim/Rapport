%%%%%%%%%%%%%%%%%%%%%%%%%%%%%%%%%%%%%%%%%%%%%%%%%%%%%%%%%%%%%%%%%%%%%%%%%%%%%%%%%%%%%%%%%%%%%
%%									section .	2.	Param\`{e}tre d’optimisation									%
%%%%%%%%%%%%%%%%%%%%%%%%%%%%%%%%%%%%%%%%%%%%%%%%%%%%%%%%%%%%%%%%%%%%%%%%%%%%%%%%%%%%%%%%%%%%%

\subsection{Paramètres d’optimisation}
 
Après une analyse du principe précèdent nous avons constaté que la distribution des états liés entrainent un nombre de calculs assez élevé pour déterminer la valeur logique de la formule sur ces états. Le regroupement de ces états améliore le temps de traitement, cela nécessite la prise en considération des objectifs fixés pour éviter la centralisation. Ainsi, les paramètres $\parametreone{}$, $\parametretwo{}$ et $\parametretree{}$ sont utilisés pour atteindre ces objectifs. La prise en compte simultanée des trois paramètres permet de minimiser à la fois le temps de traitement et assurer l’équilibrage de charge avec un nombre réduit de dupliquâts. La résolution de ce problème entraine l'utilisation de fonction objective qui permet de trouver l'espace de solutions admissibles car il peut existé une infinité de configurations respectant ces contraintes. Ainsi, différentes techniques sont offertes pour l'obtention de l'espace des solutions, dans ce contexte la technique d’équilibre de Nash combinée aux fonctions objectifs est envisageable.
\begin{description}[leftmargin=*,labelindent=1em]
	\item[$\parametreone{}$] Le paramètre de temps de traitement;
	\item [$\parametretwo{}$] Le paramètre d'équilibrage de charge entre les machines ;
	\item [$\parametretree{}$] Le paramètre de dupliquâts  entre les machines.
\end{description}


