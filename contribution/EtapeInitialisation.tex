%%%%%%%%%%%%%%%%%%%%%%%%%%%%%%%%%%%%%%%%%%%%%%%%%%%%%%%%%%%%%%%%%%%%%%%%%%%%%%%%%%%%%%%%%%%%%
%%									sub section Initialisation de $\beta ,\gamma $				%
%%%%%%%%%%%%%%%%%%%%%%%%%%%%%%%%%%%%%%%%%%%%%%%%%%%%%%%%%%%%%%%%%%%%%%%%%%%%%%%%%%%%%%%%%%%%%

\subsection{Calcul des valeurs des parametres $\parametretwo{} \; et\;\parametretree{} $ }
 
{L'initialisation des paramètres s'effectue lors de l'exécution du \mc{}. Tel que nous l'avons souligné, ces paramètres  sont initialisés à partir des états sur lesquels la formule n'est pas vérifiée. Par la suite, les états liés sont marqués par chaque état dont ils dépendent (dans l'algorithme c'est la variable \textbf{f} qui stocke cet ensemble d'états). Enfin, les états liés sont énumérés. Un état est dit lié, lorsque la formule n'est pas vérifiée sur ses successeurs directs ou indirects. Certains prédécesseurs directs ou indirects ne sont pas liés aux successeurs directs ou indirects sur lesquels la formule n'est pas vérifiée car celle-ci n'est pas vérifiée sur ces prédécesseurs. Cela motive la duplication de ces états prédécesseurs. A titre d'exemple, l'exécution de la formule  \textit{$AG(a)$} sur la structure de Kripke de la Figure(\ref{skd2}), l'état \sneuf{} est un prédécesseur indirect de l'état \s{S14}, celui-ci peut être dupliqué sur une autre machine car la vérification de la formule sur cet état est indépendante des autres états. 

Dans ce qui suit nous proposons la démarche d'initialisation des paramètres, le marquage des états liés, l'énumération des états liés et des états à dupliquer. Cette démarche est ajoutée à l'algorithme de model checking définit ci-dessus.
}
	\mysubsection{Initialisation des paramètres $\parametretwo{}$  et $\parametretree{} $ }{
Les paramètres $\parametretwo{}$ et $\parametretree{}$ sont initialisés à zéro au niveau des états qui ne comportent pas la formule à vérifier. Lorsque l'état appartient à la machine, le paramètre $\parametretree{}$ sera initialisé à $1$, cela signifie que cet état peut être dupliqué dans les machines qui le référencent. Par exemple, l'exécution de la formule  \textit{$AG(a)$} sur la structure de Kripke distribuée de la Figure(\ref{skd2}) fait qu'à l'état \sneuf{}, le paramètre $\parametretree{}$ est initialisé à 1 pour marquer la duplication de cet état. On remarque que la duplication de l'état \sneuf{}  sur la \mone{}, permet de réduire le temps de vérification de la formule, car la valeur logique de la formule sur à l'état \sneuf{} est connue, la \mtwo{} n'envoie pas alors de message concernant la valeur logique de la formule sur cet état. Le principe d'initialisation est formalisé sur l'équation (\ref{eqn3}). On obtient alors:

\begin{algorithm}[H]\label{alg1}
\SetAlgoLined
\SetKwIF{If}{ElseIf}{Else}{if}{then}{else if}{else}{endif}
\SetKwInOut{S}{$S$}
\SetKwInOut{e}{$e$}
\SetKwInOut{p}{$p$}
\SetKwInOut{ei}{$e.i$}
\SetKwInOut{lep}{$L(e,p)$}
\SetKwInOut{te}{$(type(e)$}
\SetKwInOut{ci}{$\curenti{}$}
\S{Liste des états}
\e{Un état}
\p{Une propriété}
\ei{Liste des itérations d'un état}
\lep{Liste des propriétés d'un états}
\te{Type d'un état: Border pour un état externe}
\ci{Itération courant}
\Begin{	
\ForEach{$e$ in $S$}{\label{line11}
	\If{$(\ L(e,p)=false)\  and\ (e.i=null)\ )$}{\label{line12}
			$e.\parametretwo{}=0$ \;\label{line13}
			$e.\parametretree{}=(type(e)\ne \border{})? \ 0 \ :\ 1 $\; \label{line14}
			$e.i=\{\curenti{} \}$\;\label{line15}
   }
} 
}
 \caption{Initialize Parameters}
\end{algorithm}
Les instructions décrivant l'algorithme(\ref{alg1}) sont expliquées comme suit:
\begin{description}
	\item[ligne \ref{line11} :] Parcourir l'espace d'états d'une \mi{}.
	\item[ligne \ref{line12} :] Vérifier si la formule n'est pas vérifiée sur un état et si une itération précédente n'existe pas (c'est-à-dire si $ e.i\ne null$ alors une itération l'avait déjà initialisée, mais il est à $ null$ . 
	
	Lorsque les conditions de la ligne \ref{line12} sont vérifiées alors les instructions des lignes suivantes sont exécutées:
	\item[ligne \ref{line13} :] Le paramètre $\parametretwo{}$ est initialisé à $0$(zéro).
	\item[ligne \ref{line14} :] Le paramètre $\parametretree{}$ est initialisé à $0$(zéro) lorsque l'état est un \textsl{\border{}} ( c'est-à-dire lorsqu'il appartient à une autre machine ) , sinon à $1$. Par exemple, l'exécution de la formule  \textit{$AG(a)$} sur la structure de Kripke distribuée de la Figure(\ref{skd2}). A l'état \sneuf{} : sur \mone{}  le paramètre $\parametretree{}$ est initialisé à $0$, car cet état n'appartient pas à la machine;  Par contre sur la \mtwo{} le paramètre $\parametretree{}$ est initialisé à $1$, car l'état appartient à cette machine.
	\item[ligne \ref{line15} :] L'itération courante est sauvegardée sur l'état pour éviter de reprendre l'initialisation de cet état lors d'une prochaine itération car s'il n'est pas ajouté, les états respectant les conditions de la ligne \ref{line12} sont initialisés; sans cette initialisation, cet état vérifiera les conditions de la ligne \ref{line12}. 
\end{description}
\begin{Exemple}\label{ea1}
	Après la vérification de la formule \textit{$AG(a)$} sur la structure de Kripke de la Figure (\ref{skd2}), les résultats d'initialisation sont:
\begin{description}
	\item[Itération 1]				
Dans la première itération, le paramètre $\parametretree{}$ est initialisé à $1$ sur tous les états sur lesquels la formule n'est pas vérifiée. L'attribution de la valeur 1 est expliquée au niveau de la ligne \ref{line14} de l'algorithme \ref{alg1}. Par contre le paramètre $\parametretwo{}$ est sans contrainte, alors il est à $0$. Le numéro  d'itération est enregistré sur ces états. La sauvegarde de l'itération courante est expliquée à la ligne \ref{line15} de l'algorithme \ref{alg1}.
Dans le Tableau (\ref{tii1}), on retrouve les états initialisés par les machines.   
\begin{tableth}
	\centering
	\begin{tabular}{|*{6}{c|}}
		\hline
		Id	& $\parametretwo{}$	&$\parametretree{}$&	I&	M&	T\\
		\hline
		S9	&0	&1	&1	&M2	&I\\
		\hline
		S14	&0	&1	&1	&M2	&I\\
		\hline
		S19	&0	&1	&1	&M3	&I\\
		\hline
	\end{tabular}
	\caption{Étape d'initialisation itération 1}\label{tii1}
\end{tableth}	

	\item[Itération 2]				
Après la première itération, les états sur lesquels la formule n'est pas vérifiée, et qui ont des prédécesseurs appartenant à d'autres machines sont notifiées pour prendre en compte la valeur logique de ces états. Une notification déclenche une itération. Elle permet de vérifier la formule sur l'espace d'état de la machine notifiée. Pendant l'itération, l'algorithme du model checker détecte sur certains états que la formule n'est pas vérifiée. Ces états sont liés aux états sur lesquels la notification est faite.
Une seconde itération est déclenchée sur la \mone{}, car l'état \s{S6} dispose d'un successeur sur la \mtwo{} (l'état \s{S9}). La formule n'est pas vérifiée sur cet état(détecté à Itération $1$), d'où la notification envoyée par la \mtwo{} à la \mone{}. Dans la \mone{} les paramètres $(\parametretwo{}, \parametretree{})$ de l'état \s{S9} sont initialisés à zéro. Le paramètre $\parametretree{}$ est initialisé à $0$  parce qu'il n'appartient pas à la \mtwo{}. Ainsi le Tableau (\ref{tii2}) décrit les différents états initialisés dans les machines notifiées.
      
\begin{tableth}
	\centering
	\begin{tabular}{|*{6}{c|}}
		\hline
		Id	& $\parametretwo{}$	&$\parametretree{}$&	I&	M&	T\\
		\hline
		S9&	0&	0&	2&	M1&	E\\
		\hline
		S15&0&	0&	2&	M2&	E\\
		\hline
		S10&0&	0&	2&	M1&	E\\
		\hline
	\end{tabular}
	\caption{Étape d'initialisation itération 2}\label{tii2}
\end{tableth}
	\item[Itération 3]
Après l'itération précédente, des états liés peuvent posséder des prédécesseurs sur d'autres machines. Ces dernières machines  sont notifiées. Par exemple, l'état \s{S2} est lié à l'état \s{S9}, \s{S2} possède des prédécesseurs sur la \mtwo{} et sur la \mtree{}. La notification pour cet état entraine une vérification de la formule sur chaque machine notifiée. Ainsi les paramètres$(\parametretwo{}, \parametretree{})$ de cet état sont initialisées à $0$ d'après l'algorithme d'initialisation. Le paramètre $\parametretree{}$ est initialisé à $0$  parce qu'il n'appartient pas à la machine. Le Tableau (\ref{tii3}) décrit les différents états initialisés sur  chacune des machines.    		 
\begin{tableth}
	\centering
	\begin{tabular}{|*{6}{c|}}
		\hline
		Id	& $\parametretwo{}$	&$\parametretree{}$&	I&	M&	T\\
		\hline
		S7&	0&	0&	3&	M2&	E\\
		\hline
		S2&	0&	0&	3&	M3&	E\\
		\hline
	\end{tabular}
	\caption{Étape d'initialisation itération 3}\label{tii3}
\end{tableth}
	\item[Itération 4] Après l'itération précédente, des états liés peuvent posséder des prédécesseurs sur d'autres machines. Ces dernières sont notifiées. Par exemple l'état \s{S22} est lié à l'état \s{S2}. Il possède un prédécesseur sur la \mtwo{}. La notification pour cet état, déclenche une vérification de la formule sur la \mtwo{}. Ainsi les paramètres$(\parametretwo{}, \parametretree{})$ de cet état sont initialisés à $0$, d'après l'algorithme d'initialisation. Le paramètre $\parametretree{}$ est initialisé à $0$  parce qu'il n'appartient pas à cette machine. Le Tableau (\ref{tii4}) décrit les différents états initialisés sur  chacune des machines. 			  
\begin{tableth}
	\centering
	\begin{tabular}{|*{6}{c|}}
		\hline
		Id	& $\parametretwo{}$	&$\parametretree{}$&	I&	M&	T\\
		\hline
		S22&	0&	0&	4&	M2&	E\\
		\hline
	\end{tabular}
	\caption{Étape d'initialisation itération 4}\label{tii4}
\end{tableth}
	\item[Itération 5] Après l'itération précédente, des états liés peuvent posséder des prédécesseurs sur d'autres machines. Les machines sont notifiées. Par exemple l'état \s{S11} est lié à l'état \s{S2}, des prédécesseurs sont  sur la \mone{} et sur la \mtree{}. La notification pour cet état, déclenche une vérification de la formule sur chaque cette machine. Les paramètres $(\parametretwo{}, \parametretree{})$ de cet état sont initialisés à $0$, d'après l'algorithme d'initialisation. Le paramètre $\parametretree{}$ est initialisé à $0$  parce qu'il n'appartient pas à cette machine. Le Tableau (\ref{tii5}) décrit les différents états initialisés sur chacune des machines.
\begin{tableth}
	\centering
	\begin{tabular}{|*{6}{c|}}
		\hline
		Id	& $\parametretwo{}$	&$\parametretree{}$&	I&	M&	T\\
		\hline
		S11&	0&	0&	5&	M1&	E\\
		\hline
		S11&	0&	0&	5&	M3&	E\\
		\hline
	\end{tabular}
	\caption{Étape d'initialisation itération 5}\label{tii5}
\end{tableth}	
\end{description}	
\end{Exemple}


}
\mysubsection{Marquage des états liés}{
Pendant l'exécution de l'algorithme de vérification, la valeur logique de la formule sur certains états peut dépendre des successeurs directs ou indirects, ces états sont alors liés. La formule peut être vérifiée sur les successeurs, ils peuvent appartenir à d'autres machines. Lorsque la formule n'est pas vérifiée sur un état successeur, les états liés sont alors marqués par cet état. Le principe de marquage est implémenté sur les deux algorithmes de base du model checker. Ils sont définis comme suit : 
\begin{description}
\item[Premier algorithme :] Il est défini par l'équation (\ref{eqn15}), il est utilisé lorsque la vérification concerne tous les successeurs. Ainsi Lorsque la formule n'est pas vérifiée au niveau d'un successeur, deux cas se présentent : 
	\begin{itemize}
		\item La propriété est vérifiée sur l'état, l'état est alors lié à un successeur. Les successeurs sur lesquels la formule n'est pas vérifiée sont marqués sur l'état.
		\item Lorsque la propriété n'est pas vérifiée sur un état, celui-ci peut être dupliqué soit sur la machine réalisant le traitement soit sur des machines distantes. Les états successeurs sur lesquels la formule n'est pas vérifiée sont enregistrés dans l'ensemble \s{limit}, cela signifie que cet état est indépendant de ses successeurs et peut donc être dupliqué sur des machines distantes. Sur l'exemple (\ref{ea1}) de l'Algorithme(\ref{alg1}), la valeur logique de la formule à l'état \s{S9} est indépendante des états \s{S14} et \s{S7}. L'état \s{S9} peut alors être dupliqué sur les machines qui possèdent ses prédécesseurs.
	\end{itemize}  
   
\item[Deuxième algorithme :] Il est défini par l'équation(\ref{eqn16}). Il se différencie du premier algorithme par la vérification de la formule sur les successeurs. Il vérifie l'existence de la propriété sur un successeur. 
\end{description}
En plus des conditions des algorithmes, l'existence de l'itération courante est vérifiée. Cette condition permet de marquer les états de cette itération. Ce principe est formalisé comme suit:\\
\begin{algorithm}[H]\label{alg2}
\SetAlgoLined
\SetKwIF{If}{ElseIf}{Else}{if}{then}{else if}{else}{endif}
$AXf \longleftarrow \emptyset$\; \label{line21}
\ForEach{$e$ in $S$}{\label{line22}
	\uIf{$\left( (\forall e' \; in\; succ(e) \subseteq \left(T^+(f)\cup U^+(f)\right))and(e \in \{T^+(f)\cup U^+(f)\}  )\right)$}{\label{line23}
		$AXf\longleftarrow AXf\cup \left\{e\right\}$\;\label{line24} 
	}
	\Else{\label{line25}
		\uIf{$\left( e\ \;\;in\;\; T \right)$}{\label{line26}
			\ForEach{$e'$ in $succ(e)$}{\label{line27}
				\If{$\left( \curenti{}\;\; in \;\; e'.i \right)$}{\label{line28}
					$e.f\longleftarrow e.f\cup e'.f$\;\label{line29}
					$e.i\longleftarrow e.i\cup \{  \curenti{}\}$\;\label{line210}			
				}			
	   		}\label{endfor27}
   		}\label{line2111}
   		\Else{\label{line211}
   			\ForEach{$e'$ in $succ(e)$}{\label{line212}
				\If{$\left( \curenti{}\;\;in \;\; e'.i \right)$}{\label{line213}
					$e.limit\longleftarrow e.limit\cup e'.f$\;\label{line214}
					$e.fn \longleftarrow e.fn\cup \left\{  \curenti{} \right\}$\; \label{line215}				
				}			
	   		}   		
   		}
	}	
} 
 \caption{Mark linked states formula AX}
\end{algorithm}

Les instructions décrivant l'Algorithme(\ref{alg2}) sont expliquées comme suit:
\begin{description}
	\item[ligne \ref{line21} :] L'ensemble \s{AXf} contient les états sur lesquels la formule est vérifiée ou peut être vérifiée. Cet ensemble est initialisé par l'instruction de cette ligne.
	\item[ligne \ref{line22} :] L'instruction de cette ligne permet de parcourir les états stockés sur la \mi{}
	\item[ligne \ref{line23} :] L'instruction de cette ligne vérifie la formule sur les successeurs, si elle est vérifiée, alors l'instruction de la ligne(\ref{line24}) est exécutée. 
	\item[ligne \ref{line24} :] L'instruction de cette ligne stocke l'état dans l'ensemble \s{AXf}.
	\item[ligne \ref{line25} :] Lorsque la formule n'est pas vérifiée sur l'état ou ses successeurs, les instructions  des lignes suivantes sont alors exécutées. 
	\item[ligne \ref{line26} :] L'instruction de cette ligne vérifie la formule sur l'état. Lorsqu'elle est vérifiée, les instruction des lignes( \ref{line27} à \ref{endfor27}) sont alors exécutées. La ligne \ref{line27} permet de parcourir les successeurs de l'état pour vérifier  l'itération courante grâce à l'instruction de la ligne \ref{line28}. Une fois qu'elle est présente, les instructions suivantes sont exécutées:
	\begin{description}
	\item[ligne \ref{line29} :] L'instruction de cette ligne permet de récupérer les états dont ils dépendent. 
	\item[ligne \ref{line210} :] L'instruction de cette ligne enregistre l'itération courante sur l'état, car un état peut dépendre de plusieurs états de différentes itérations. 
	\end{description}
	\item[ligne \ref{line211} :] Lorsque la formule n'est pas vérifiée sur l'état,ses successeurs sont parcourus grâce à l'instruction de la ligne \ref{line212}. Lors de ce parcours, l'itération courante est vérifiée par l'instruction de la ligne \ref{line213}.\\ Lorsque l'itération courante est présente, les états successeurs sur lesquels la formule n'est pas vérifiée sont enregistrés dans la variable \s{limit} de l'état grâce à l'instruction de la ligne \ref{line214}. La sauvegarde de ces états permet d'éviter le déplacement de cet état, étant donné que sa duplication permet à l'algorithme de connaitre la valeur logique de ces prédécesseurs sans que la machine concernée ne soit notifiée.
	
\end{description}

\begin{algorithm}[H]\label{alg3}
\SetAlgoLined
\SetKwIF{If}{ElseIf}{Else}{if}{then}{else if}{else}{endif}
$EXf \longleftarrow \emptyset$\; \label{line31}
\ForEach{$e$ in $S$}{\label{line32}
	\uIf{$\left( \exists e'  in succ(e) \subseteq \left(T^+(f)\cup U^+(f)\right)\right)$}{\label{line33}
		$EXf\longleftarrow EXf\cup \left\{e\right\}$\; \label{line34}
	}
	\Else{\label{line35}
		\uIf{$\left( e\ \;\;in\;\; T \right)$}{\label{line36}
			\ForEach{$e'$ in $succ(e)$}{\label{line37}
				\If{$\left( \curenti{}\;\;in\;\; e'.i \right)$}{\label{line38}
					$e.f\longleftarrow e.f\cup e'.f$\;\label{line39}
					$e.i\longleftarrow e.i\cup \{  \curenti{}\}$\;\label{line310}				
				}			
	   		}
   		}
   		\Else{\label{line311}
   			\ForEach{$e'$ in $succ(e)$}{\label{line312}
				\If{$\left( \curenti{}\;\; in\;\; e'.i \right)$}{\label{line313}
					$e.limit\longleftarrow e.limit\cup e'.f$\;\label{line314}
					$e.fn \longleftarrow e.fn\cup \left\{  \curenti{} \right\}$\; \label{line315}				
				}			
	   		}   		
   		}
	}\label{line320}	
} 
 \caption{Mark linked states formula EX}
\end{algorithm}

Les instructions décrivant l'Algorithme(\ref{alg3}) sont similaires à celle de l'Algorithme(\ref{alg2}), la différence se trouve au niveau de la vérification de la formule sur les successeurs. Pour l'Algorithme(\ref{alg3}), lorsque la formule est vérifiée ou peut être vérifiée sur un successeur, l'instruction de la ligne \ref{line34} est alors exécutée, sinon les instructions de la ligne \ref{line35} à la ligne \ref{line320} sont exécutées en fonction de la vérification de la formule comme expliqué au niveau de  l'Algorithme(\ref{alg2}).

\begin{Exemple}\label{ea2}
	L'application de l'algorithme de model checker de la formule \textit{$AG(a)$} sur la structure de Kripke de la Figure (\ref{skd2}) donne les états marqués à chaque itération comme suit:
\begin{description}
	\item[Itération 1] Dans la première itération, les états initialisés dans l'exemple(\ref{ea1}) sont utilisés pour déterminer les états liés. Le model checker utilise l'Algorithme(\ref{alg2}) pour vérifier la formule \textit{$AG(a)$}, cet algorithme détecte sur les états(\s{S12},\s{S10}) qu'ils sont liés à l'état \s{S14}, car la formule n'est pas vérifiée sur l'état \s{S14}, par contre la formule est vérifiée sur ces états. La vérification de la formule sur les successeurs directs de l'état \s{S14}, ce dernier est un successeur direct de \s{S12}, ou indirect	s de l'état \s{S14}, ce dernier est un successeur indirect de l'état \s{S10}, cela montre qu'elle n'est pas vérifiée sur chacun de ces états. l'état \s{S14} est marqué dans l'ensemble de dépendance des états liés. L'état \s{S9} est un prédécesseur indirect de l'état \s{S14}, ce dernier est marqué sur l'ensemble \s{limit} de l'état \s{S9} car la formule n'est pas vérifiée sur état \s{S9}. Le Tableau (\ref{tim1}) représente les différents états marqués par chacune des machines pendant cette itération.   
\begin{tableth}
	\centering
	\begin{tabular}{|*{7}{c|}}
		\hline
		Id&		T&			F&	I&	limit&	fn&		M\\
		\hline
		S9&		I&$\emptyset$&$\{	1\}$&	$\{S14\}$&	$\{S14\}$&	M2\\
		\hline
		S10&	I&	$\{S14\}$&$\{	1\}$&$\emptyset$& $\emptyset$ &M2\\
		\hline
		S12&	I&	$\{S14\}$&$\{	1\}$&$\emptyset$&$\emptyset$ &	M2\\
		\hline
		S15&	I&	$\{S19\}$&$\{	1\}$&$\emptyset$&$\emptyset$ &	M3\\
		\hline
		S16&	I&	$\{S19\}$&$\{	1\}$&$\emptyset$&$\emptyset$ &	M3\\
		\hline
		S17&	I&	$\{S19\}$&$\{	1\}$&$\emptyset$&$\emptyset$ &	M3\\
		\hline
		S18&	I&	$\{S19\}$&$\{	1\}$&$\emptyset$&$\emptyset$ &	M3\\		
		\hline
	\end{tabular}
	\caption{Étape de marquage: itération 1}\label{tim1}
\end{tableth}
\\\\	
	\item[Itération 2] Dans cette itération les résultats obtenues dans l'exemple(\ref{ea1}) à l'itération 2 sont utilisés pour déterminer les états liés aux états notifiés(dans l'exemple(\ref{ea1}) une notification provoque l'itération 2 sur la \mone{}, cette notification concerne l'état \s{S9}). Pendant la vérification de la formule, l'algorithme détecte la valeur logique de la formule au niveau des états(\s{S1}, \s{S2}, \s{S3}, \s{S4},\s{S5}, \s{S6}) sur la \mone{}. La valeur logique de cette formule sur ces états dépend de la valeur logique de la formule  sur l'état \s{S9}. Ces états sont des prédécesseurs directs ou indirects de l'état  \s{S9}. Cet état est rajouté à l'ensemble  de dépendance, l'itération courante est ainsi marquée. 
Le Tableau (\ref{tim2}) représente les différents états marqués pendant l'itération 2 de chacune des machines.  
	\begin{tableth}
	\centering
	\begin{tabular}{|*{7}{c|}}
		\hline
		Id&		T&			F&	I&	limit&	fn&		M\\
		\hline
		S1&	I&	$\{S9,S10\}$&	$\{1,2\}$&	$\emptyset$& $\emptyset$&		M1\\ \hline
		S2&	I&	$\{S9,S10\}$&	$\{1,2\}$&	$\emptyset$& $\emptyset$&		M1\\ \hline
		S3&	I&	$\{S9,S10\}$&	$\{1,2\}$&	$\emptyset$& $\emptyset$&		M1\\ \hline
		S5&	I&	$\{S9,S10\}$&	$\{1,2\}$&	$\emptyset$& $\emptyset$&		M1\\ \hline
		S6&	I&	$\{S9\}$	   &	$\{2\}$&	$\emptyset$& $\emptyset$&		M1\\ \hline
		S7&	I&	$\{S9\}$    &	$\{2\}$&	$\emptyset$& $\emptyset$&		M1\\ \hline
		S8&	I&	$\{S10\}$   &	$\{2\}$&	$\emptyset$& $\emptyset$&		M1\\ \hline
		S9&	E&	$\emptyset$&	$\{2\}$&	$\emptyset$& $\emptyset$&		M1\\ \hline
		S10&E&	$\emptyset$&	$\{2\}$&	$\emptyset$& $\emptyset$&		M1\\ \hline
		S14&I&	$\emptyset$&	$\{1\}$&	$\{S15\}$   & $\{S15\}$&			M2\\ \hline
		S15&E&	$\emptyset$&	$\{2\}$&	$\emptyset$& $\emptyset$&		M2\\				
		\hline
	\end{tabular}
	\caption{Étape de marquage: itération 2}\label{tim2}
\end{tableth}
	\item[Itération 3] Dans cette itération les résultats obtenues dans l'exemple(\ref{ea1}) à l'itération 3 sont utilisés pour déterminer les états liés aux états notifiés de l'exemple(\ref{ea1}. Une notification provoque l'itération 3 sur la \mtree{}, cette notification concerne l'état \s{S2}). Pendant la vérification, l'algorithme détecte la valeur logique de l'état(\s{S20}) sur la \mtree{}, elle dépend de l'état \s{S2}. Ce dernier est ajouté à l'ensemble  de dépendance de l'état \s{S20}. Par contre l'état \s{S19} est un prédécesseur indirect de l'état \s{S2}, il est ajouté à l'ensemble \s{limit} de l'état (\s{S19}, car la formule n'est pas vérifiée sur cet état l'à. Toute fois l'itération courante est marquée sur l'état \s{S19}. 
le Tableau (\ref{tim3}) représente les différents états marqués pendant l'itération 3 de chacune des machines. 
	\begin{tableth}
	\centering
	\begin{tabular}{|*{7}{c|}}
		\hline
		Id&		T&			F&	I&	limit&	fn&		M\\
		\hline
		S9&	I&	$\emptyset$&	$\{1\}$&	$\{S7\}$   &	$\{S7\}$&	M2\\ \hline
		S7&	E&	$\emptyset$&	$\{3\}$&	$\emptyset$& $\emptyset$&	M2\\ \hline
		S2&	E&	$\emptyset$&	$\{3\}$&$\emptyset$& $\emptyset$	&	M3\\ \hline
		S19&I&	$\{S19\}$&	$\{1\}$&$\{S2\}$   & $\{S2\}$	&	M3\\ \hline
		S20&I&	$\{S2\}$&	$\{3\}$&$\emptyset$& $\emptyset$&	M3\\ \hline
		S22&I&	$\{S2\}$&	$\{3\}$&$\emptyset$& $\emptyset$&	M3\\ \hline
		
	\end{tabular}
	\caption{Étape de marquage: itération 3}\label{tim3}
\end{tableth}
\\\\\\\\\\
	\item[Itération 4] Dans cette itération les résultats obtenues dans l'exemple(\ref{ea1}) à l'itération 4 sont utilisés pour déterminer les états liés aux états notifiés dans l'exemple(\ref{ea1}. Une notification provoque l'itération 4 sur la \mtwo{}, la notification concerne l'état \s{S22}). Le principe de marquage est similaire à l'itération précédente, le Tableau (\ref{tim4}) représente alors les différents états marqués pendant l'itération 4 de chacune des machines. 
	\begin{tableth}
	\centering
	\begin{tabular}{|*{7}{c|}}
		\hline
		Id&		T&			F&	I&	limit&	fn&		M\\
		\hline
		S11&	I&	$\{S22\}$	&	$\{4\}$			&$\emptyset$	&$\emptyset$	&M2\\ \hline
		S13&	I&	$\{S22\}$	&	$\{4\}$			&$\emptyset$	&$\emptyset$	&M2\\ \hline
		S14&	I&	$\emptyset$	&$\emptyset$	&	$\{S15,S22\}$	&$\{S15,S22\}$	&M2\\ \hline
		S22&	E&	$\emptyset$	&	$\{4\}$			&$\emptyset$	&$\emptyset$	&M2\\ \hline
		
	\end{tabular}
	\caption{Étape de marquage: itération 4}\label{tim4}
\end{tableth}\\

	\item[Itération 5] Dans cette itération les résultats obtenues dans l'exemple(\ref{ea1}) à l'itération 5 sont utilisés pour déterminer les états liés aux états notifiés de l'exemple(\ref{ea1}. Une notification provoque l'itération 5 sur la \mtree{}, la notification concerne l'état \s{11}). Ainsi l'algorithme détecte la valeur logique des états(\s{S15}, \s{S16}, \s{S18}) sur la \mtree{}, elle dépend de l'état \s{S11}. Le principe de marquage est similaire à l'itération 2, on remarque sur ces états l'effet de plusieurs itérations, cela s'explique par le fait que ces états dépendent de plusieurs autres états. Le Tableau (\ref{tim5}) représente les différents états marqués pendant l'itération 5 de chacune des machines.
	\begin{tableth}
	\centering
	\begin{tabular}{|*{7}{c|}}
		\hline
		Id&		T&			F&	I&	limit&	fn&		M\\
		\hline
		S1&		I&$\emptyset$&		$\{1 ,5\}$&	$\{S7\}$&	$\{S7\}$&		M2\\ \hline
		S2&		I&$\emptyset$&		$\{1 ,5\}$&$\emptyset$&$\emptyset$&		M2\\ \hline
		S11&	I&$\emptyset$&		$\{5\}$	&$\emptyset$&$\emptyset$&		M2\\ \hline
		S11& 	E&$\emptyset$&		$\{5\}$&$\emptyset$&$\emptyset$&		M3\\ \hline
		S15&	I&	$\{S19,s11\}$&	$\{1 ,5\}$&$\emptyset$&$\emptyset$&		M3\\ \hline
		S16&	I&	$\{S19,s11\}$&	$\{1 ,5\}$&$\emptyset$&$\emptyset$&		M3\\ \hline
		S18&	I&	$\{S19,s11\}$&	$\{1 ,5\}$&$\emptyset$&$\emptyset$&		M3\\ \hline
	\end{tabular}
	\caption{Étape de marquage: itération 5}\label{tim5}
\end{tableth}\\

\end{description}	
\end{Exemple}\break
} 
\mysubsection{Énumération des états liés}{
Après le marquage des états, on obtient sur les états liés les états dont ils dépendent. Ainsi pour chaque état dépendant, les états qui lui sont liés sont énumérés, cela permet d'attribuer des valeurs aux paramètres $(\parametretwo{}$ et $\parametretree{})$. L'attribution des valeurs peut concerner les états appartenant à la machine locale ou appartenant aux machines distantes. Ces valeurs permettent d'effectuer un redéploiement des états sur la bonne machine. Selon le type de chaque état la technique d'énumération est la suivante:
\begin{description}
\item[\ee{} :] Les états externes(\s{border}) sont des états appartenant aux machines distantes. Lorsque la formule n'est pas vérifiée sur ces états, le nombre d'états liés sont recensés. La valeur calculée est attribuée au paramètre $\parametretwo{}$ de l'état \s{border}. 
Dans l'exemple(\ref{ea2}), la valeur du paramètre $\parametretwo{}$ est égale à $2$ à l'état \s{S22} de la \mtwo{}.
 Ainsi la valeur du paramètre $\parametretree{}$ correspond au nombre d'états pour lesquels l'état \s{border} a été inséré dans leurs ensembles respectifs \s{limit}. Dans le cas des états externes, la duplication de ces états sur une machine donne une structure de Kripke incohérente (la structure obtenue présente des états dupliqués sans aucuns prédécesseurs sur la machine elle même, ce modèle est alors incohérent vis à vis d'une structure de Kripke). Pour obtenir une structure cohérente il est envisageable de déplacer ces états sur d'autres machines tout en laissant les duplicatas sur la machine local. Pour ce faire le paramètre$\parametrefour{}$ est utilisé pour stocker le nombres de ces états à la place du paramètre $\parametretree{}$. La valeur de ce paramètre est calculée similairement  à celle du paramètre $\parametretree{}$.
 L'application de ce principe sur l'exemple(\ref{ea2}) permet d'attribuer la valeur $1$ au paramètre $\parametrefour{}$ à l'état \s{S22} de la \mtwo{}, car cet état est stocké dans l'ensemble \s{limit} de l'état \s{S14}.  
 
\item[\ei{} :]  Les états internes sont des états appartenant à la machine locale. Lorsque ces états disposent des prédécesseurs directs sur d'autres machines distantes, les valeurs des paramètres sont alors calculés. La valeur du paramètre $\parametretwo{}$ correspond aux nombres des successeurs directs et indirects présents entre les état marqués sur l'ensemble de dépendance (la variable \s{\textbf{f}}) de l'état, ce nombre est incrémenté de $1$ car l'état dépend de ces états l'à. Par contre la valeur du paramètre $\parametretree{}$  correspond aux nombre des états internes présents sur l'ensemble de dépendance.
  L'application de ce principe sur l'exemple(\ref{ea2}) permet d'obtenir la valeur $2$ pour le paramètre $\parametretwo{}$ et $0$ pour le paramètre $\parametretwo{}$ à l'état \s{S7} de la \mtwo{}.  
\end{description}

Ces principes sont formalisés comme suit:


\begin{function}
	\setcounter{AlgoLine}{0}
	\caption{getSucess($e:state$): List of State}  
  \SetKwInOut{S}{$ElementSucc$}
  \SetKwInOut{e}{$e$}
  \SetKwInOut{ei}{$e.f$}
  \SetKwInOut{te}{$succ(e)$}
  \S{Liste des états Successeur}
  \e{Un état}
  \ei{Liste que dépendant un état}
  \te{Successeur d'un état: Border pour un état externe}  
 
  	\Begin{
  $ElementSucc	\longleftarrow\emptyset$\;
  \ForEach{$e'\in succ(e)$}{ \label{linef1}
   		\If{$((e'\nexists e.f)and((e'.f\cap e.f)\ne\emptyset)$ }{\label{linef2}
    		 	$ElementSucc\longleftarrow ElementSucc\cup\{e'\}\cup \printlcs{e'}$\;\label{linef3}
    		}
    	}
    	
    \KwRet ElementSucc \;\label{linef4}}
\end{function}

Les instructions  de la fonction \s{getSucc} permettent de récupérer les successeurs directs et indirects entre un état et les états dont-il dépend. La ligne \ref{linef1} permet de parcourir les successeurs directs de l'état afin de vérifier leur dépendance avec lui. Cela est faite à la ligne \ref{linef2}.  Lorsque le successeur est en dépendance avec l'état encours, il est ajouté à l'ensemble des successeurs suivi de l'appel de la fonction \s{getSucc} pour le traitement transitif de ses successeurs. Le résultat retourné par cette fonction  est ajouté  à l'ensemble des successeurs. Ces instructions sont exécutées à la ligne \ref{linef3}. Ainsi, la fonction retournera touts les successeurs de l'état en cours à la fin de son exécution.  
\\

\begin{algorithm}[H]\label{alg4}
\SetAlgoLined
\SetKwIF{If}{ElseIf}{Else}{if}{then}{else if}{else}{endif}

\Begin{
\ForEach{$e$ in $S$}{\label{line42}
	\uIf{$\left((e  \in border(M))and (e\in F) and (e.i==\curenti{})\right)$}{\label{line43}
		\ForEach{$e'$ in $S-\{ e\}$}{\label{line44}
			\uIf{$\left((e  \in e.f)\right)$}{\label{line45}
				$e.\parametretwo{} \longleftarrow e.\parametretwo{} +1$\;\label{line46}
			}
			\uElseIf{$(e \in e'.limit )$}{
				$e.\parametretwo{} \longleftarrow e.\parametretwo{} +1$\;\label{line47}
				$e.\parametrefour{} \longleftarrow e.\parametrefour{} +1$\;\label{line48}				
			}				
		}
	}
	\uElseIf{$\left((e  \in Notifier(M))and (e\in T) and (\curenti{}\in e.i)\right)$}{\label{line49}
		$e.\parametretwo{} \longleftarrow Count(getSucc(e))+1$\;\label{line410}
		\ForEach{$e" \in e.f$}{\label{line411}
					\If{$\left( (Type(e") \ne Border AND  e"!=e\right)$}{\label{line415}
						$e.\parametretree{} \longleftarrow e.\parametretree{} +1$\;
					}
		}\label{line416}
	}	}
} 
 \caption{Count parameters values}
\end{algorithm}

Les instructions décrivant l'Algorithme \ref{alg4} sont expliquées comme suit:

\begin{description}
	\item[ligne \ref{line42} :]  Cette ligne présente une instruction de parcours des états stockés sur la \mi{}.
	\item[ligne \ref{line43} :]  L'instruction de cette ligne vérifie que l'état est \s{externe}, la formule n'est pas vérifiée à cet état, et l'itération courante est marquée sur l'état. La dernière condition permet de calculer les valeurs des paramètres d'un état de l'itération courante vérifiant les deux premières conditions. Lorsque ces conditions sont vérifiées, l'espace d'états est alors parcouru par l'instruction de la ligne \ref{line44}, et calcule les valeurs des paramètres. Pour chaque état est vérifiée:
	
	\begin{itemize}
	\item Si l'état externe est présent dans l'ensemble de dépendance de l'état encours, le paramètre $\parametretwo{}$ de l'état externe est alors incrémenté (ligne \ref{line46}) car cet état est lié à l'état externe.
	\item Sinon, si l'état externe est présent dans l'ensemble \s{limit} de cet état, les paramètres $\parametretwo{}$ et $  \parametrefour{}$ sont alors incrémentés (ligne \ref{line47} et ligne \ref{line48}) car cet état peut être déplacer en laissant un dupliquât sur la machine effectuant le traitement.  
	\end{itemize}
	
	\item[ligne \ref{line49} :] L'instruction de cette ligne vérifie sur l'état l'existence des prédécesseurs directs sur des autres machines distantes, la dépendance des successeurs et la présence de l'itération courante. Lorsque ces conditions sont vérifiées, les successeurs  entre les états liés sont alors calculés, car la valeur du paramètre $\parametretwo{}$ correspond au nombre des successeurs entre l'état et les états dont-il dépend (ligne \ref{line41}). Par contre la valeur du paramètre $\parametretree{}$ correspond au nombre des états dont-il dépend appartenant à la machine locale, les instructions sont de la ligne \ref{line411} jusqu'à la ligne \ref{line416}.
\end{description}

\begin{Exemple}\label{ea3}
	L'application de l'Algorithme(\ref{alg4}) permet de déterminer pour chacun des états de l'exemple(\ref{ea1}), les valeurs des paramètres à travers les résultats de l'exemple(\ref{ea2}). Les valeurs calculées sont: 
	
\begin{description}
	\item[Itération 1] La première itération comptabilise pour chaque état \s{Border} ou \s{Notifer} de l'exemple(\ref{ea1}) à l'itération 1 le nombre des états  qui lui sont liés. Ce nombre est affecté au paramètre  $\parametretwo{}$, par exemple dans l'exemple(\ref{ea2}) aucun état dépend de l'état \s{S9} sur la \mtwo{}. Cependant, il peut-être dupliqué sur la \mone{} car la formule n'est pas vérifiée sur cet état, la valeur du paramètre $\parametretree{}$  est alors égale à $1$. Ainsi, les valeurs des paramètres de l'état \s{S14} ne sont pas calculés car il ne possède pas de prédécesseurs sur chacune des autres machines.
Le Tableau (\ref{tie1}) représente les valeurs des paramètres  des états.	
\begin{tableth}
	\centering
	\begin{tabular}{|*{7}{c|}}
		\hline
		Id&$\parametretwo{}$&	$\parametretree{}$	&$\parametrefour{}$ &	I&	M&	T\\ \hline
		S9&		0&	1	&0&	1&	M2&	Notifier\\ \hline
		S10&	2&	1	&0&	1&	M2&	Notifier\\ \hline
		S15&	3&	1	&0&	1&	M3&	Notifier\\ \hline		
	\end{tabular}
	\caption{Calcul des valeurs des parametres: itération 1}\label{tie1}
\end{tableth}
\item[Itération 2] Dans cette itération les valeurs de l'état \s{S15} sont expliquées car le calcul des paramètres des autres états est similaire à l'itération précédente. Ainsi, dans l'exemple(\ref{ea2}) il n'existe pas d'état dépendant de l'état \s{S15} sur la \mtwo{}, par contre sur l'état \s{S14} il est inséré dans l'ensemble \s{limit}. Ainsi les valeurs des paramètres $\parametretwo{}$ et $\parametrefour{}$  sont égales à $1$ car l'état \s{S14} peut être déplacé sur la \mtwo{}. Cet état est donc dupliqué sur la machine locale. 
Le Tableau (\ref{tie2}) représente les valeurs des paramètres  des états.
\begin{tableth}
	\centering
	\begin{tabular}{|*{7}{c|}}
		\hline
		Id&$\parametretwo{}$&	$\parametretree{}$	&$\parametrefour{}$ &	I&	M&	T\\ \hline
		S2&		6&	0&	&	2&	M1&	Notifier\\ \hline
		S7&		2&	0&	&	2&	M1&	Notifier\\ \hline
		S9&		6&	0&	&	2&	M1&	Border\\ \hline
		S15&	1&	0&	1&	2&	M2&	Border\\ \hline		
	\end{tabular}
	\caption{Calcul des valeurs des parametres: itération 2}\label{tie2}
\end{tableth}
\item[Itération 3] Dans cette itération, la démarche pour l'état \s{S2} est similaire à celle de l'état \s{S15} auquel est associé la démarche de l'itération 1 pour calculer les états liés à cet état. Par contre les autres états sont similaires à l'itération 1. Le Tableau (\ref{tie3}) représente les valeurs des paramètres  des états. 
\begin{tableth}
	\centering
	\begin{tabular}{|*{7}{c|}}
		\hline
		Id&$\parametretwo{}$&	$\parametretree{}$	&$\parametrefour{}$ &	I&	M&	T\\ \hline
		S2&		3&	0&	1&	3&	M3&	Border\\ \hline
		S7&		1&	0&	&	3&	M2&	Border\\ \hline
		S22&	1&	0&	&	3&	M3&	Notifier\\ \hline	
	\end{tabular}
	\caption{Calcul des valeurs des parametres: itération 3}\label{tie3}
\end{tableth}

\item[Itération 4] La démarche de cette itération est similaire à l'itération précédente. Le Tableau (\ref{tie4}) représente les valeurs des paramètres  des états.  
\begin{tableth}
	\centering
	\begin{tabular}{|*{7}{c|}}
		\hline
		Id&$\parametretwo{}$&	$\parametretree{}$	&$\parametrefour{}$ &	I&	M&	T\\ \hline
		S11&	2&	0&	0&	4&	M2&	Notifier\\ \hline
		S2&		3&	0&	1&	4&	M2&	Border\\ \hline
	\end{tabular}
	\caption{Calcul des valeurs des parametres: itération 4}\label{tie4}
\end{tableth}

\item[Itération 5] La démarche de cette itération est similaire à l'itération 3. Le Tableau (\ref{tie5}) représente les valeurs des paramètres  des états.
\begin{tableth}
	\centering
	\begin{tabular}{|*{7}{c|}}
		\hline
		Id&$\parametretwo{}$&	$\parametretree{}$	&$\parametrefour{}$ &	I&	M&	T\\ \hline
		S11&	2&	0&	0&	5&	M1&	Border\\ \hline
		S11&	3&	0&	0&	5&	M3&	Border\\ \hline
		S15&	4&	1&	0&	5&	M3&	Notifier\\ \hline
	\end{tabular}
	\caption{Calcul des valeurs des parametres: itération 5}\label{tie5}
\end{tableth}

\end{description}	
\end{Exemple}
}

\mysubsection{Échange des valeurs des paramètres}{
Après le calcul des valeurs des paramètres, ces valeurs sont envoyées aux machines concernées par l'état. A la réception des valeurs, ces dernières sont stockées sur l'état concerné. Ce prince est formalisé comme suit:
\begin{procedure}[H]
	\setcounter{AlgoLine}{0}	
\caption{SendValueCalculate ($id,\parametretwo{},\parametretree{},\parametrefour{},f$)}

  	\Begin{  
\nl  		\ForEach{$e\in \{Notifier(S)\cup Border(S)\}$}{ \label{linesend1}
\nl   		\If{$(\curenti{} \in e.i)$ }{\label{linesend2}
\nl   			$F=(type(e)==Notifier) ? e : \emptyset$\;\label{linesend3}\\
\nl    		 	\ForEach{$m \in e.site $}{\label{linesend4}
\nl   				$send(e.id,e.\parametretwo{},e.\parametretree{},F,e.\parametrefour{},i)\;\;  to \;\;m.id )$\;
    			}\label{linesend5}
    		}
    	}  
   }  	

\end{procedure}

L'envoi des valeurs concernent les états appartenant aux machines distantes sur lesquels la formule n'est pas vérifiée, et les états locale la formule n'est pas vérifiée qui possèdent des prédécesseurs directs sur d'autres machines. Le parcours de ces états est fait par l'instruction de la ligne \ref{linesend1} de la procédure d'envoi. L'envoi des valeurs concerne les états de l'itération courante, car les valeurs sont échangées après chaque itération. La vérification de la présence de l'itération courante est effectuée à la ligne \ref{linesend2}. Lorsqu'elle est pressente, les valeurs calculées sont envoyées à toutes les machines concernées par cet état. Cela est fait de la ligne \ref{linesend3} jusqu'à la ligne \ref{linesend5}.

\begin{Exemple}\label{ea4}
	Le processus d'envoi des valeurs obtenues dans l'exemple(\ref{ea3}), celui-ci montre que les valeurs de certains états ne sont pas envoyées car ces états ne disposent pas de prédécesseurs directs sur des machines distantes.  Les valeurs envoyées pour chaque état sont décrites comme suit:
	\begin{description}
	\item[Itération 1]
		\begin{itemize}
		\item La \mtwo{} envoie les informations: $id =s9$; $\parametretwo{}=0$; $\parametretree{}=1$; $\parametrefour{}=0$; $f =\{S9\}$; $i=2$ (le paramètre i désigne l'identité de la machine) à la \mone{}.
		\item La \mtwo{} envoie les informations: $id =s10$; $\parametretwo{}=2$; $\parametretree{}=1$; $\parametrefour{}=1$;\\$f =\{S10\}$; $i=2$  à la \mone{}.
		\item La \mtree{} envoie les informations: $id =s15$; $\parametretwo{}=3$; $\parametretree{}=0$; $\parametrefour{}=1$;\\$f =\{s19\}$; $i=3$	 à la \mtwo{}.
		\end{itemize}
	\item[Itération 2]
		\begin{itemize}
			\item La \mone{} envoie les informations: $id =s9$; $\parametretwo{} =6$; $\parametretree{} =0$; $\parametrefour{}=0$;\\$f =\emptyset$, $i=1$ à la \mtwo{}.
			\item  La \mone{} envoie les informations: $id =s10$; $\parametretwo{}=5$; $\parametretree{}=0$; $\parametrefour{}=0$;\\$f =\emptyset$; $i=1$ à la \mtwo{}.
			\item  La \mone{} envoie les informations: $id =s7$ ;$\parametretwo{}=2$; $\parametretree{}=0$; $\parametrefour{}=0$;\\$f =\{s9\}$; $i=1$ à la \mtwo{}.
			\item  La \mone{} envoie les informations: $id =s2$; $\parametretwo{}=5$; $\parametretree{}=0$; $\parametrefour{}=0$;\\$f =\{s9,s10\}$; $i=1$ à la \mtree{}.
			\item  La \mtwo{} envoie les informations: $id =s15$; $\parametretwo{}=1$; $\parametretree{}=0$; $\parametrefour{}=1$;\\$f =\emptyset$; $i=2$ à la \mtree{}.
			\end{itemize}	
\item[Itération 3]
		\begin{itemize}
			\item  La \mtwo{} envoie les informations: $id =s7$; $\parametretwo{}=1$; $\parametretree{}=0$; $\parametrefour{}=1$;\\$f =\emptyset$; $i=1$ à la \mone{}.
			\item  La \mtree{} envoie les informations: $id =s2$; $\parametretwo{}=3$; $\parametretree{}=0$; $\parametrefour{}=1$;\\$f =\emptyset$; $i=3$ à la \mone{}.
			\item  La \mtree{} envoie les informations: $id =s22$; $\parametretwo{}=1$; $\parametretree{}=0$; $\parametrefour{}=0$;\\$f =\{s2\}$; $i=3$ à la \mtwo{}.
			\end{itemize}
		\item[Itération 4]
		\begin{itemize}
			\item  La \mtwo{} envoie les informations: $id =s11$; $\parametretwo{}=2$; $\parametretree{}=0$; $\parametrefour{}=0$;\\$f=\{s22\}$; $i=2$ à la \mone{}.
			\item  La \mtwo{} envoie les informations: $id =s11$; $\parametretwo{}=2$; $\parametretree{}=0$; $\parametrefour{}=0$;\\$f=\{s22\}$; $i=2$ à la \mtree{}.
			\item  La \mtwo{} envoie les informations: $id =s22$; $\parametretwo{}$=2; $\parametretree{}=0$; $\parametrefour{}=0$;\\$f=\emptyset$; $i=2$ à la \mtwo{}.
			\end{itemize}
			\item[Itération 5]
		\begin{itemize}
		\item  La \mone{} envoie les informations: $id =s2$; $\parametretwo{}=5$; $\parametretree{}=0$; $\parametrefour{}=0$;\\$f=\{s9,s10,s11\}$; i=1 à la \mtree{}.
			\item  La \mone{} envoie les informations: $id =s11$; $\parametretwo{}=2$; $\parametretree{}=0$; $\parametrefour{}=0$;\\$f=\emptyset$; $i=1$ à la \mtwo{}.
			\item  La \mtree{} envoie les informations: $id =s11$; $\parametretwo{}=3$; $\parametretree{}=0$; $\parametrefour{}$=0;\\$f =\{s22\}$; $i=2$ à la \mtwo{}.
			\item La \mtree{} envoie les informations: $id =s15$; $\parametretwo{}=3$; $\parametretree{}=0$; $\parametrefour{}=1$;\\$f=\{s19,s11\}$; $i=3$	 à la \mtwo{}.
			\end{itemize}	
	\end{description}

\end{Exemple}

\begin{procedure}[H]
\setcounter{AlgoLine}{0}	
\SetAlgoLined
\LinesNumbered
  \caption{Reception de: $(id,\parametretwo{},\parametretree{},\parametrefour{},f,i)$ par la machine j}
  	\Begin{
  \nl\ForEach{$e\in \; S$}{ \label{linereceve1}
  \nl 		\If{$(e.id==id )$ }{\label{linereceve2}
  \nl 			$x.site[i].\parametretwo{} \longleftarrow \parametretwo{}$\;\\
  \nl 			$x.site[i].\parametretree{} \longleftarrow \parametretree{}$ \;\\
  \nl 			$x.site[i].\parametrefour{} \longleftarrow \parametrefour{}$\;\\
  \nl 			$x.site[i].f\longleftarrow f$\;
   			}
    		} 	
    	}
\end{procedure}

Les instructions décrivant la procédure de réception des valeurs, permettent de stocker les valeurs reçues sur l'état concerné. lorsque la machine reçoit les valeurs, l'état consterné est recherché (la ligne \ref{linereceve1} présente l'instruction du parcours des états). Une fois l'état trouvé, les valeurs reçues sont stockées sur l'état dans l'ensemble \s{site}, il contient les informations des machines sur cet état, l'indice de cet liste représente la clé de la machine. 

\begin{Exemple}\label{ea4}
	En appliquant le processus de réception les valeurs reçues sont stockées sur l'état concerné. Ainsi, nous remarquerons que les valeurs de certains états sont mises à jours avec les itérations.  Les valeurs reçues pour chacun des états sont les suivantes:
	
	\begin{description}
	\item[Itération 1]
		\begin{itemize}
		\item A la réception des valeurs: $id =s9$; $\parametretwo{}=0$; $\parametretree{}=1$; $\parametrefour{}=0$; $f =\{S9\}$; $i=2$ par la \mone{}. Ces dernières sont stockées à l'indice \s{i} de la variable  \s{site} de l'état \s{S9}.
		
		\item A la réception des valeurs: $id =s10$; $\parametretwo{}=2$; $\parametretree{}=1$; $\parametrefour{}=1$; $f =\{S10\}$; $i=2$ par la \mone{}. Ces dernières sont stockées à l'indice \s{i} de la variable  \s{site} de l'état \s{S10}.
		
		\item A la réception des valeurs: $id =s15$; $\parametretwo{}=3$; $\parametretree{}=0$; $\parametrefour{}=1$; $f =\{s19\}$; $i=3$ par la \mtwo{}. Ces dernières sont stockées à l'indice \s{i} de la variable  \s{site} de l'état \s{S15}.
		\end{itemize}
	\item[Itération 2]
		\begin{itemize}
			\item A la réception des valeurs: $id =s9$; $\parametretwo{} =6$; $\parametretree{} =0$; $\parametrefour{}=0$; $f =\emptyset$, $i=1$ par la \mtwo{}. Ces dernières sont stockées à l'indice \s{i} de la variable  \s{site} de l'état \s{S9}.
			
			\item  A la réception des valeurs: $id =s10$; $\parametretwo{}=5$; $\parametretree{}=0$; $\parametrefour{}=0$; $f =\emptyset$; $i=1$ par la \mtwo{}. Ces dernières sont stockées à l'indice \s{i} de la variable  \s{site} de l'état \s{S10}.
			\item  A la réception des valeurs: $id =s7$; $\parametretwo{}=2$; $\parametretree{}=0$; $\parametrefour{}=0$; $f =\{s9\}$; $i=1$ par la \mtwo{}. Ces dernières sont stockées à l'indice \s{i} de la variable  \s{site} de l'état \s{S7}.
			\item  A la réception des valeurs: $id =s2$; $\parametretwo{}=5$; $\parametretree{}=0$; $\parametrefour{}=0$; $f =\{s9,s10\}$; $i=1$ par la \mtree{}. Ces dernières sont stockées à l'indice \s{i} de la variable  \s{site} de l'état \s{S2}.
			\item  A la réception des valeurs: $id =s15$; $\parametretwo{}=1$; $\parametretree{}=0$; $\parametrefour{}=1$; $f =\emptyset$; $i=2$ par la \mtree{}. Ces dernières sont stockées à l'indice \s{i} de la variable  \s{site} de l'état \s{S15}.
			\end{itemize}	
\item[Itération 3]
		\begin{itemize}
			\item  A la réception des valeurs: $id =s7$; $\parametretwo{}=1$; $\parametretree{}=0$; $\parametrefour{}=1$; $f =\emptyset$; $i=1$ par la \mone{}. Ces dernières sont stockées à l'indice \s{i} de la variable  \s{site} de l'état \s{S7}.
			\item  A la réception des valeurs: $id =s2$; $\parametretwo{}=3$; $\parametretree{}=0$; $\parametrefour{}=1$; $f =\emptyset$; $i=3$ par la \mone{}. Ces dernières sont stockées à l'indice \s{i} de la variable  \s{site} de l'état \s{S2}.
			\item  A la réception des valeurs: $id =s22$; $\parametretwo{}=1$; $\parametretree{}=0$; $\parametrefour{}=0$; $f =\{s2\}$; $i=3$ par la \mtwo{}. Ces dernières sont stockées à l'indice \s{i} de la variable  \s{site} de l'état \s{S22}.
			\end{itemize}
		\item[Itération 4]
		\begin{itemize}
			\item  A la réception des valeurs: $id =s11$; $\parametretwo{}=2$; $\parametretree{}=0$; $\parametrefour{}=0$; $f =\{s22\}$; $i=2$ par la \mone{}. Ces dernières sont stockées à l'indice \s{i} de la variable  \s{site} de l'état \s{S11}.
			\item  A la réception des valeurs: $id =s11$; $\parametretwo{}=2$; $\parametretree{}=0$; $\parametrefour{}=0$; $f =\{s22\}$; $i=2$ par la \mtree{}. Ces dernières sont stockées à l'indice \s{i} de la variable  \s{site} de l'état \s{S11}.
			\item  A la réception des valeurs: $id =s22$; $\parametretwo{}=2$; $\parametretree{}=0$; $\parametrefour{}=0$; $f =\emptyset$; $i=2$ par la \mtwo{}. Ces dernières sont stockées à l'indice \s{i} de la variable  \s{site} de l'état \s{S22}.
			\end{itemize}
			\item[Itération 5]
		\begin{itemize}
		\item  A la réception des valeurs: $id =s2$; $\parametretwo{}=5$; $\parametretree{}=0$; $\parametrefour{}=0$; $f =\{s9,s10,s11\}$; $i=1$ par la \mtree{}. Ces dernières sont stockées à l'indice \s{i} de la variable  \s{site} de l'état \s{S2}.
			\item  A la réception des valeurs: $id =s11$; $\parametretwo{}=2$; $\parametretree{}=0$;$\parametrefour{}=0$; $f =\emptyset$; $i=1$ par la \mtwo{}. Ces dernières sont stockées à l'indice \s{i} de la variable  \s{site} de l'état \s{S11}.
			\item  A la réception des valeurs: $id =s11$; $\parametretwo{}=3$; $\parametretree{}=0$; $\parametrefour{}=0$; $f =\{s22\}$; $i=2$ par la \mtwo{}. Ces dernières sont stockées à l'indice \s{i} de la variable  \s{site} de l'état \s{S11}.
			\item A la réception des valeurs: id =s15;$\parametretwo{}$=3 ;$\parametretree{}$=0 ;$\parametrefour{}=1$; $f =\{s19,s11\}$; $i=3$	 par la \mtwo{}. Ces dernières sont stockées à l'indice \s{i} de la variable  \s{site} de l'état \s{S15}.
			\end{itemize}	
	\end{description}
\end{Exemple}
}