\section{Logique temporelle arborescente (CTL)}
La logique temporelle définie des opérateurs permettant de \s{lier} des variables propositionnelles \emph{p} aux instants \emph{t}. Par exemple, une assertion liée au comportement d'un programme telle que \s{après exécution d'une instruction i, le système se bloque}, les actions de cette assertion s'exécutent suivant un axe de temps : à l'instant \emph{t}, exécution de l'instruction \emph{i}, et à \emph{$t + 1$} blocage du système. La logique du temps arborescent permet de modéliser les expressions du passé et du futur \citep{Bakker1983}, \citep{Clarke1981}, \citep{Emerson1983}.\\

La logique du temps arborescent (CTL) «Computation Tree Logic (en aglais)» permet d'exprimer des propriétés portant sur les arbres d'exécution (issus de l'état initial) du programme. Elle modélise des expressions du passé et du futur à partir d'un état du système. Une expressions CTL exprime des propriétés relatives à l'arbre d'exécution grâce à une syntaxe et une sémantique.

\subsection{Syntaxe de CTL}
Le langage des formules de CTL sont les formules d'état définies inductivement par:
\begin{itemize}
	\item  $ \forall p \in AP$, p est une formule d'état;
	\item  Si p, q sont des formules d'état, alors $p \vee q$ et $\neg p$ sont des formules d'état;
	\item Si p est une formule de chemin, alors \emph{$E_p$} et \emph{Ap} sont des formules de chemin;
	\item Si p et q sont des formules d'état, alors \emph{$X_p$} et \emph{pUq} sont des formules d'état.
\end{itemize}

\subsection{Sémantique de CTL}
La sémantique de CTL est interprétée dans une structure de Kripke. Les états sont décorés par les propositions atomiques ${\displaystyle p}$.\\
Soit $q_0$ un état de S. La sémantique de CTL est définie inductivement par:
\begin{itemize}
	\item  $p \in AP$, $q_0 \models p \Leftrightarrow p \in L(q_0 )$
	\item $q_0 \models p\vee q \Leftrightarrow q_0 \models p\; ou\; q_0 \models q$
	\item $q_0 \models \neg p \Leftrightarrow not(q_0 \models p)$
	\item  $q_0 \models Ep \Leftrightarrow  \exists \delta =q_0 …q_n$ telle que $\delta \models p$ (p formules de chemins)
	\item $q_0 \models Ap \Leftrightarrow \forall \delta =q_0 …q_n$ on a $\delta \models p$ (p formules de chemins)
	\item$ \delta \models Xp \Leftrightarrow \delta _1 \models p$ (p formules d'état)
	\item $ \delta \models pUq \Leftrightarrow \exists j \geq 0, \delta _j \models q\; et (\forall k <j, \delta _k \models p)$ (p et q formules d'état)
\end{itemize}
les opérateurs temporels sont définies par:
\begin{itemize}
	\item A (All), E (Exist): quantifications sur toutes les exécutions possibles à partir de l'état courant
	\item X (next), F (eventually), G (always),U précédent directement  A, E
	\item $\neg , \vee, \wedge$
\end{itemize}

Pour illustrer l'utilisation de la logique temporelle arborescence dans la formalisation des problèmes pratiques, nous donnons l'exemple du publiphone qui consiste à modéliser les actions qu'un utilisateur exécute pendant l'opération de tentative de téléphoner jusqu'à la fin de la communication. La procédure téléphonée est la suivante :
\begin{itemize}
	\item L'utilisateur doit décrocher le téléphone;
	\item Le système affiche insérer une carte;
	\item L'utilisateur insère une carte;
	\item Le système affiche le nombre d'unités restantes;
	\item L'utilisateur compose son numéro de téléphone avec la possibilité de le corriger pour obtenir le   bon numéro dans un délai de 2 secondes;
	\item L'utilisateur communique avec son correspondant, tant qu'il lui reste des unités. – Si la carte est épuisée, l'utilisateur a 10 secondes pour la changer;
	\item Lorsque la communication est terminée, l'utilisateur raccroche le téléphone;
	\item Le système affiche retirer la carte. – L'utilisateur retire sa carte;
\end{itemize}
Ensuite, formalisons les propositions atomiques :
\begin{itemize}
	\item u\_décrocher : l'utilisateur décroche le téléphone
	\item sys\_affiche(message) : le système affiche le message
	\item u\_insérer\_carte : l'utilisateur insère une carte
	\item u\_compser\_no : l'utilisateur compose le numéros
	\item u\_correction\_no : l'utilisateur corrige le numéros
	\item numéros\_correct : le numéros est correct
	\item u\_communiquer : l'utilisateur communique avec son correspondant
	\item u\_changer\_carte\_vide : l'utilisateur change sa carte qui est épuisée
	\item u\_raccrocher : l'utilisateur raccroche le téléphone
	\item u\_retirer\_carte : l'utilisateur retire sa carte
	\item ctl\_téléphoner : établit la clause de l'activité téléphoner
\end{itemize}

Enfin, on obtient la formalisation en CTL:\\

$u\_decrocher \; \wedge\; u\_inserer\_carte \wedge 	\\
AX [(u\_composer\_no\; \wedge\; EF (u\_reprise\_sur\_erreur)) \wedge \\
AX ((u\_communiquer \;\wedge\; EF (changer\_carte\_vide)) U \\
(u\_raccrocher \;\wedge \; u\_retirer\_carte))] \Rightarrow \; ctl\_telephoner$