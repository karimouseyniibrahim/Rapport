%%%%%%%%%%%%%%%%%%%%%%%%%%%%%%%%%%%%%%%%%%%%%%%%%%%%%%%%%%%%%%%%%%%%%%%%%%%%%%%%%%%%%%%%%%%%%
%%									section.	1.	Notions mathématiques											%
%%%%%%%%%%%%%%%%%%%%%%%%%%%%%%%%%%%%%%%%%%%%%%%%%%%%%%%%%%%%%%%%%%%%%%%%%%%%%%%%%%%%%%%%%%%%%

\section{Notions mathématiques}
Cette section rappelle quelques définitions mathématiques.
\begin{definition}[Ensemble]
Un ensemble désigne une collection d'éléments, la notation utilisé pour décrire un ensemble est $S =\{s_1, . . ., s_n\}$. Les ensembles utilisés dans ce document sont finis et leur taille est notée $\mid S \mid$. On note $\emptyset$ l'ensemble vide.
\end{definition}

\begin{definition}[Cardinal d'un ensemble]
Soit X un ensemble fini d'éléments. Le cardinal de l'ensemble X est égal au nombre éléments de X, et on le note card(X).
\end{definition}

\begin{definition}[Partition]
Soit un ensemble S quelconque. Un ensemble P de sous-ensembles de S est appelé une partition de S si:
\begin{itemize}
	\item Aucun élément de P n'est vide;
	\item L'union des éléments de P est égale à S;
	\item Les éléments de P sont deux à deux disjoints.
\end{itemize}
 Les éléments de P sont appelés les parties de la partition P. Le cardinal de la partition P est alors le nombre de parties de P.
\end{definition}

Le partitionnement de graphe fait partie des problèmes d'optimisation combinatoire, qui est une branche des mathématiques discrètes. Ce dernier parfois appelées mathématiques finies, sont l'étude des structures mathématiques où la notion de continuité est absente. Les objets étudiés en mathématiques discrètes (tels que les entier relatifs, les graphes simples et les énoncés en logique) sont des ensembles dénombrables \citep{Norman1989}. Dans le cadre des mathématiques discrètes, un ensemble dénombrable (ensembles qui ont la même cardinalité que les sous-ensembles des nombres naturels, y compris les nombres rationnels mais pas les nombres réels) peut aussi être appelé ensemble discret.